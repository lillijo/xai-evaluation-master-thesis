% ---------------------------------------------------
% ----- Main document of the template
% ----- for Bachelor-, Master thesis and class papers
% ---------------------------------------------------
%  Created by C. Müller-Birn on 2012-08-17, CC-BY-SA 3.0.
%  Last upadte: C. Müller-Birn 2015-11-27
%  Freie Universität Berlin, Institute of Computer Science, Human Centered Computing. 

\documentclass[pdftex,a4paper,12pt,DIV=calc,BCOR=5mm,openany,bigheadings,titlepage, halfparskip-]{report}   

\setlength\parskip{10pt}
% ----- weitere Optionen 
%draft,			% Entwurfsmodus zum Anzeigen zu leerer/voller Boxen 
%DIV=calc
%DIV12,			% Seitengröße (siehe Koma Skript Dokumentation !) 
%BCOR5mm,		% Zusätzlicher Rand auf der Innenseite 
%twoside,		% Seitenränder werden an doppelseitig angepasst 
%fleqn,			% Formeln werden linksbündig (und nicht zentriert) angezeigt 
%titlepage,		% Titel wird in einer 'titlepage' Umgebung gesetzt 
%bigheadings,	% Große Überschriften (normal, small-headings) 
%halfparskip-	% Absatz wird nicht eingerückt, dafür aber um eine halbe Zeile nach unten gerückt
%
%---------------------------------------------------
%----- Packages
%---------------------------------------------------
%
\usepackage[T1]{fontenc} 
\usepackage[utf8]{inputenc}
\usepackage[english]{babel}  
\usepackage{ae}   
\usepackage{enumitem}
\usepackage{todonotes}
\usepackage{fancyhdr} % Define simple headings 
\usepackage{xcolor}
\usepackage{color}
\usepackage{mdframed}
\usepackage{url}
\usepackage{listings}
%\usepackage{vmargin} % Adjust margins in a simple way
%
\usepackage{amsmath}			% MUSS vor fontspec geladen werden
\usepackage{mathtools}			% modifiziert amsmath
\usepackage{amssymb}			% mathematische symbole, für \ceckmarks
\usepackage{amsthm}				% für proof
\usepackage{mathrsfs}	

\usepackage{graphicx}  
\usepackage{hyperref} 
\usepackage[noabbrev, nameinlink, capitalise]{cleveref}
% turn all your internal references into hyperlinks
%\usepackage[pdfstartview=FitH,pdftitle={<<Titel der Arbeit>>}, pdfauthor={<<Autor>>}, pdfkeywords={<<Schlüsselwörter>>}, pdfsubject={<<Titel der Arbeit>>}, colorlinks=true, linkcolor=black, citecolor=black, urlcolor=black, hypertexnames=false, bookmarksnumbered=true, bookmarksopen=true, pdfborder = {0 0 0}]{hyperref}
%
\usepackage{tikz, ifthen}
%paragraph settings
%\setlength{\parskip}{1em}
% table settings 
\usepackage{booktabs}  
\usepackage{tabularx}  
\usepackage{rotating}
\usepackage{longtable}
%\usepackage{lscape}
\usepackage{multirow} %multi row
%\usepackage{rotating} %for rotating table
\usepackage{pdfpages}
\usepackage{float}
\usepackage{times}
\usepackage{cite}
\usepackage{natbib}
\bibliographystyle{abbrvnat}
\setcitestyle{authoryear,open={(},close={)}}
\usepackage[section]{placeins}

%---------------------------------------------------
%----- PDF and document setup
%---------------------------------------------------
%
\hypersetup{
	pdftitle={Are We Explaining the Data or the Model? Concept-Based Methods and Their Fidelity in Presence of Spurious Features Under a Causal Lense},  % please, add the title of your thesis
    pdfauthor={Lilli Joppien},   % please, add your name
    pdfsubject={Master thesis, Institute of Computer Science, Freie Universität Berlin}, % please, select the type of this document
    pdfstartview={FitH},    % fits the width of the page to the window
    pdfnewwindow=true, 		% links in new window
    colorlinks=false,  		% false: boxed links; true: colored links
    linkcolor=red,          % color of internal links
    citecolor=green,        % color of links to bibliography
    filecolor=magenta,      % color of file links
    urlcolor=cyan           % color of external links
}
\renewcommand*{\chapterheadstartvskip}{\vspace*{0cm}}
\renewcommand*{\chapterheadendvskip}{\vspace{0.5cm}}

% 
%---------------------------------------------------
%----- Customize page size
%---------------------------------------------------
\usepackage[top=2cm,right=3cm,bottom=4cm,left=3cm]{geometry}    
%
%---------------------------------------------------
%----- Customize header and footer\pagestyle{fancy} 
%---------------------------------------------------


\fancyhf{}  % delete all existing header formating
\pagestyle{fancy}
%\fancyhead[RE]{\nouppercase{\leftmark}}  % Chapter in the right on evens
%\fancyhead[LO]{\nouppercase{\rightmark}}
%\renewcommand{\chaptermark}[1]{ % adapt the shown chapter name: show it in lower case and with chapter number 
%\markboth{\thechapter.\ #1}{}}   

\fancyhead[L]{\nouppercase{\rightmark} }
\fancyhead[R]{\nouppercase{\leftmark}}

\renewcommand{\headrulewidth}{0pt} % remove lines from header
\renewcommand{\footrulewidth}{0pt} % remove lines from header

% independence sign _||_
\newcommand\independent{\protect\mathpalette{\protect\independenT}{\perp}}
\def\independenT#1#2{\mathrel{\rlap{$#1#2$}\mkern2mu{#1#2}}}
  
% declare my measures as variables to have text above and below them 
\DeclareMathOperator*{\MLC}{MLC}
\DeclareMathOperator*{\MAC}{MAC}
\DeclareMathOperator*{\RMA}{RMA}
\DeclareMathOperator*{\RE}{Re}
\DeclareMathOperator*{\RRA}{RRA}
\DeclareMathOperator*{\SPF}{SPF}
\DeclareMathOperator*{\YSPF}{y-SPF}
\DeclareMathOperator*{\PG}{PG}
\DeclareMathOperator*{\sort}{sort}

\newcommand{\changelocaltocdepth}[1]{%
  \addtocontents{toc}{\protect\setcounter{tocdepth}{#1}}%
  \setcounter{tocdepth}{#1}%
}

%\fancyfoot{} % delete all existing footer formating
%\fancyfoot[LE,RO]{\thepage} % put page number on the left on even page and right on odd page
%
%---------------------------------------------------      
%----- Settings for word separation  
%---------------------------------------------------      
% Help for separation (from package babel, section 22)):
% In german package the following hints are additionally available:
% "- = an explicit hyphen sign, allowing hyphenation in the rest of the word
% "| = disable ligature at this position. (e.g., Schaf"|fell)
% "~ = for a compound word mark without a breakpoint (e.g., bergauf und "~ab)
% "= = for a compound word mark with a breakpoint, allowing hyphenation in the composing words
% "" = like "-, but producing no hyphen sign (e.g., und/""oder)
%
% Describe separation hints here:
\hyphenation{
% Pro-to-koll-in-stan-zen
% Ma-na-ge-ment  Netz-werk-ele-men-ten
% Netz-werk Netz-werk-re-ser-vie-rung
% Netz-werk-adap-ter Fein-ju-stier-ung
% Da-ten-strom-spe-zi-fi-ka-tion Pa-ket-rumpf
% Kon-troll-in-stanz
}
%
%---------------------------------------------------
%----- Restricting including files   
%---------------------------------------------------
% Only files listed here will be included in the PDF document!
% In order to only partially translate the document, for example for bug-fixing, 
% it might be useful to comment out some of the documents.
\includeonly{
title,
declaration,
abstract_en,
abstract_de,
acknowledgements,
introduction,
background,
problem_setting,
method0_scm,
method1_crp,
method2_m0_m1,
method3_m2_mac,
method4_m2_rest,
method5_comparison,
results,
discussion,
appendix
}

%%%%%%%%%%%%%%%%%%%%%%%%%%%%%%%%%%%%%%%%%%%%%%%%%%%%%%
% The content part of the document starts here! %%
%%%%%%%%%%%%%%%%%%%%%%%%%%%%%%%%%%%%%%%%%%%%%%%%%%%%%%

\begin{document}
%---------------------------------------------------
%----- Listing and color definition   
%---------------------------------------------------
\definecolor{red}{rgb}{.8,.1,.2}
\definecolor{blue}{rgb}{.2,.3,.7}
\definecolor{lightyellow}{rgb}{1.,1.,.97}
\definecolor{gray}{rgb}{.7,.7,.7}
\definecolor{darkgreen}{rgb}{0,.5,.1}
\definecolor{darkyellow}{rgb}{1.,.7,.3}
\lstloadlanguages{C++,[Objective]C,Python}

\definecolor{codegreen}{rgb}{0,0.6,0}
\definecolor{codegray}{rgb}{0.5,0.5,0.5}
\definecolor{codepurple}{rgb}{0.58,0,0.82}
\definecolor{backcolour}{rgb}{0.95,0.95,0.92}
\definecolor{graybg}{rgb}{0.85,0.85,0.82}

\lstdefinestyle{mystyle}{
    backgroundcolor=\color{backcolour},   
    commentstyle=\color{codegreen},
    keywordstyle=\color{magenta},
    numberstyle=\tiny\color{codepurple},
    stringstyle=\color{codepurple},
    basicstyle=\ttfamily\footnotesize,
    breakatwhitespace=false,         
    breaklines=true,                 
    captionpos=b,                    
    keepspaces=true,                 
    %numbers=left,                    
    %numbersep=5pt,                  
    showspaces=false,                
    showstringspaces=false,
    showtabs=false,                  
    tabsize=2
}
% TIKZ STUFF
\usetikzlibrary{arrows.meta,arrows}
\usetikzlibrary{shapes}

\lstset{style=mystyle}
%---------------------------------------------------
%----- Title and declaration   
%---------------------------------------------------
\pagenumbering{gobble}
% ---------------------------------------------------
% ----- Title page of the template
% ----- for Bachelor-, Master thesis and class papers
% ---------------------------------------------------
%  Created by C. Müller-Birn on 2012-08-17, CC-BY-SA 3.0.
%  Freie Universität Berlin, Institute of Computer Science, Human Centered Computing. 
%

\title{
{\small Masterarbeit}\\
{\small Climate Informatics TU Berlin / Causal Inference Group DLR Jena}\\
[7ex]
{Are We Explaining \\ the Data or the Model?} \\
[1ex]
{\LARGE Concept-Based Methods and
Their Fidelity in Presence of Spurious Features Under a Causal Lense.}}

\author{
{\emph{\normalsize Lilli Joppien}}\\
[18ex]   
{\normalsize Betreuer*innen: Oana-Iuliana Popescu, Simon Bing} \\
{\normalsize Erstgutachter: Prof. Dr. Jakob Runge} \\
{\normalsize Zweitgutachter: Prof. Dr. Tim Landgraf (oder Georges Montavon?) }}
\vspace{6ex}
\date{\normalsize Berlin, \today}
\maketitle


%---------------------------------------------------
%----- Change word wrapping if it is too annoying 
%---------------------------------------------------
%\emergencystretch 3em
%\raggedright
%\sloppy
\hyphenpenalty=4000
\tolerance=2000
\raggedbottom
%---------------------------------------------------
%----- Abstracts in English and German   
%---------------------------------------------------

\subsection*{Abstract}
\begin{itemize}
    \color{red}
    \item The abstract must not contain references, as it may be used without the main article. It is acceptable, although not common, to identify work by author, abbreviation or RFC number. (For example, "Our algorithm is based upon the work by Smith and Wesson.")
    \item Avoid use of "in this paper" in the abstract. What other paper would you be talking about here?
    \item Avoid general motivation in the abstract. You do not have to justify the importance of the Internet or explain what QoS is.
    \item Highlight not just the problem, but also the principal results. Many people read abstracts and then decide whether to bother with the rest of the paper.
    \item Since the abstract will be used by search engines, be sure that terms that identify your work are found there. In particular, the name of any protocol or system developed and the general area ("quality of service", "protocol verification", "service creation environment") should be contained in the abstract.
    \item Avoid equations and math. Exceptions: Your paper proposes E = m c 2.
\end{itemize}

\subsubsection*{Motivation}
\begin{itemize}
    \item explainable AI shows great progress in visualizing how neural networks see/decide
    \item however there have been many criticisms and some argue that the XAI methods don't show what is actually seen by the NN and rely more on hyperparameters or the data itself.
    \item For example, it is known that some attribution methods do not react well to constant vector shifts in the data which do not affect prediction.
    \item it is especially unclear how the network deals with causal constructs: is there a difference between how it displays cause and effect, can it find important interactions between 2 variables or find spurious correlations?
    \item we want to identify how the ground truth biasedness of a dataset interacts with the biasedness of the model and the biasedness of the explanation
    \item for general attribution methods it has been shown that heatmaps can be misleading. If the spurious feature has any correlation with the core feature, it will have importance assigned. Often, the spurious feature comes as a watermark which is easy to identify. Consequently its importance can be overestimated when looking at a general heatmap of an image.
    \item Looking at individual concepts with their relevances and specific heatmaps has the potential to identify which of the features (core or spurious) is actually most relevant.
\end{itemize}

\subsubsection*{Problem Statement}
\begin{itemize}
    \item investigate the example of CRP, a recent method which takes the popular Layer-Wise Relevance Propagation to the next level, by producing conditional attributions for neurons or sets of neurons coined "concepts"
    \item find out, whether the heatmaps or relevances produced by this algorithm have a connection either to the causal ground truth of data or the "causal pathways" in the NN
\end{itemize}

\subsubsection*{Approach}
\begin{itemize}
    \item for validation purposes very simple disentangling dataset DSPRITES
    \item introduce "causal" biases into dataset, by adding small watermark not uniformly to certain images
    \item use a very small neural network, which seems to learn the bias strongly (check for accuracy)
    \item as preliminary experiment check, if the bias is strongly visible in the data: if the heatmaps/crp hierarchies produced on average for the watermarked/unwatermarked subsets differ strongly
    \item \textit{do causality lol}
\end{itemize}

\subsubsection*{Results}
\begin{itemize}
    \item does CRP succeed in identifying the true biasedness of the model
    \item what do we want to explain
    \item does this result generalize for other attribution methods, data, SCMs?
\end{itemize}

\subsubsection*{Conclusions}
\begin{itemize}
    \item found a new benchmark measure to combat the critique about the robustness and fidelity of especially concept-based methods.
    \item from that new method a way to enrich or improve those methods arises
    \item it is important to look at explanations in a more causal light because that is what they are ought do be doing
    \item what else needs to be done especially
\end{itemize}

% ---------------------------------------------------
% ----- Abstract (German) of the template
% ----- for Bachelor-, Master thesis and class papers
% ---------------------------------------------------
%  Created by C. Müller-Birn on 2012-08-17, CC-BY-SA 3.0.
%  Freie Universität Berlin, Institute of Computer Science, Human Centered Computing. 
%
\pagestyle{empty}

\subsection*{Zusammenfassung}

Hier ist eine Deutsche Zusammenfassung die so noch nicht existiert, um zu testen ob ich auch sachen zu overleaf schicken kann.

\cleardoublepage  
\subsection*{Acknowledgements}

First, I would like to thank my supervisor Prof. Dr. Jakob Runge, who agreed to my not very climate-informatics or causal-inferency topic and helped me sharpen my scientific expression and notation again and again.
A huge thank you to my advisors Oana-Iuliana Popescu and Simon Markus Bing who have supported me through many frustrations and blockades. 
They not only complemented each other perfectly and were always there for questions but also gave me fresh perspectives on my problems and motivated me to go the extra mile.
I owe a special thanks to the researchers at the Explainable Artificial Intelligence group of the Frauenhofer Heinrich-Hertz-Institute, especially Dr. Sebastian Lapuschkin, Maximilian Dreyer and Reduan Achtibat. They not only in many sessions accompanied me in defining a research question but also were always available for technical advice on their methods. This thesis would not have been possible if Prof. Dr. Wojciech Samek had not been interested in a joint project with the climate informatics group at TU. From that group I would also like to thank Dr. Jonas Wahl and Dr. Urmi Ninad for their contribution to the initial discussion. 
My many hours in the library were lightened up by the support and good times spent with Hannah, who helped me navigate my thesis-life balance. Of course I also extend big thank yous to my family and all my other friends who kept me sane. 
Last, but not least, I want to thank my partner Duncan for his unconditional support, be it by taking any other worries out of my hands or by answering all my English questions. 
I could not have done this without you.

                                          
%---------------------------------------------------
%----- Directories   
%---------------------------------------------------

\frontmatter 
\pagenumbering{roman}

\tableofcontents
%\setcounter{tocdepth}{3}   % reduce the included sections in the table of content

\listoffigures
%\listoftables

%---------------------------------------------------
%----- Main part
%---------------------------------------------------
\mainmatter
\pagenumbering{arabic} 
\pagestyle{fancy} 

%\include{preface} 

\chapter{Introduction}\label{chapter:introduction}

\section{Motivation and Context}
With the explosion in popularity of deep neural networks, the need to explain such black-box models has risen too. Recently, explainable AI (XAI) methods have been scrutinized more quantitatively. Although there is still no consensus on what exactly makes a \textit{good explanation}, numerous potential metrics for the faithfulness, robustness and interpretability of explanations have been made \citep{Nauta2023}. A promising branch of research seems to be the evaluation of XAI with causal methods \citep{Moraffah2020a}. After all, explanations are, or at least should be, intrinsically causal constructs \citep{Woodward2004, Halpern2005, Schoelkopf2019}.

Local attribution methods explain a decision of a neural network by attributing importance to local features such as pixels in a computer vision task. 
Due to the stronger focus on evaluation of XAI, local attribution methods have come under criticism, amongst others, for their general lack of sensitivity to the model they are trying to explain \citep{Adebayo2018, Karimi2023}. Also, some methods' class-insensitivity \citep{Sixt2020} and their failure in presence of suppressor variables and in the ``limit of simplicity'' \citep{Wilming2023} have been examined. Other authors have criticized the lack of comparative user-guided evaluation of these explanation methods \citep{Rong2023}. 
As local attribution methods, especially back-propagation methods, create visually compelling results through attribution maps, they have still become a staple for many AI practitioners, especially for computer vision tasks.  

The recent attribution method Concept-Relevance-Propagation (CRP) introduced in \cite{Achtibat2022} has been developed for a more fine-grained explanation of a neural network's decisions. Additionally to producing one attribution map explaining the overall prediction output such as Layer-wise Relevance Propagation (LRP) \citep{Bach2015}, each neuron, i.e., ``concept'', in some hidden layer of the network gets assigned a relevance and its own saliency map. In addition to the saliency maps, the relevance scores for each of the concepts also act as a metric to maximize when searching representative samples. According to the authors, through this more detailed and global explanation one can not only understand where a model sees the most relevant features, but also what features are relevant in this area. This is why they also call their method a ``glocal'' method. Their assumption is that the deeper layers of models represent concepts which are human-understandable and therefore aid in the explanation of what the model predicts. Discovering learned artifacts or concepts in training data which are spuriously correlated to the class to be predicted but have no actual causal relationship is a principal goal of explanation methods. CRP's parent LRP has been shown to uncover these \textit{Clever-Hans} or spurious features in many cases, often with additional analysis related to CRP \citep{Lapuschkin2019, Anders2022}. It is therefore not surprising that the authors of CRP also especially assess their method for the task of discovering these artifacts. 

In qualitative examinations as well as a small human subject study \citep{Achtibat2023}, they find that Clever-Hans features are easier to identify with the concept-based approach and their relevance better determined in comparison to other local attribution methods. Nevertheless there have been very few quantitative or more formal evaluations of concept-based methods like CRP.
Especially the task of determining spurious correlations has not been assessed quantitatively for these explanation methods. 
Finding whether a background or spuriously correlated feature is biasing a model is one of the most important applications of XAI, especially in applications where \textit{fairness}, \textit{robustness} and \textit{out-of-distribution prediction} are necessary. Yet the ability to accurately assign \textit{relative} importance to features in conformity with what the model encodes has not been thoroughly analyzed and formalized for concept-based methods in our knowledge. 

Here, we therefore evaluate another desirable characteristic of XAI for the method of CRP: that its sensitivity to spurious feature importance closely follows that of the network it is trying to explain. The general faithfulness or sensitivity to a model has been studied with varying methods, most often creating perturbations in the input or model at the \textit{important} feature, and testing the decline in accuracy. Some recent works have also evaluated an explanation's feature importance when spuriously correlated features are present \citep{Yang2019,Kim2018,Parafita2019,Reimers2020,Singla2022}. Arguably, high sensitivity to the model is especially crucial when identifying and quantifying spurious correlations. Confirming that the true or \textit{core} feature has some importance is not as convincing of an explanation if the importance of other features can not be rigorously analyzed and compared to it \citep{Singla2022}. \cite{Arras2022} question whether a feature can or should have 100 percent of the importance assigned to it, which is even more unclear for spurious features. Their approach to expect all importance to be within the boundary of a certain object is an evaluation of coherence with human understanding rather than fidelity to the underlying model, but the two are often conflated as \cite{Nauta2023} describe. 

In the presence of a correlated feature, where an importance of 0 percent for the model might be desirable but is often unattainable, it is even less clear which importance really to expect for the model both for core and spurious features. The question becomes yet harder to answer when the Clever-Hans feature is not spatially separable from the core feature.
We believe that looking at how explanation and model react to a growing correlation between a core and Clever-Hans feature in a continuous and relative fashion might help understand the relationship better. 

This approach is further justified by the correlated and often contaminated data we see in real applications of AI. 
In a realistic setting it is nearly impossible for a model to not have learned at least some spurious correlations. A more philosophical question here, is whether we should really expect a machine learning model to completely ignore them, if even humans are prone to false visual perception like optical illusions. In image classification datasets it is not necessarily clear which causal relationships exist between the class and what is seen in the picture. While ignoring artifacts such as watermarks seems desirable, some features might have no direct causal relationship with the target feature but learning them is not per-se wrong. 
For example, a model should be able to identify a cow even if it is standing in an unexpected environment like a beach but it seems still reasonable to be more ``alert'' to seeing cows when a green pasture is depicted. 

So in accordance with ideas of \textit{causality} it seems infeasible to aim for completely unbiased models, as they would need to have all knowable and unknowable knowledge of the universe to not predict \textit{out-of-distribution}. 
Instead, one perhaps needs to define a measurable threshold of correlation, which is acceptable for a specific type of spuriously correlated feature. 
Embedding the task into the \textit{causality} framework therefore helps to formalize how Clever-Hans or other forms of spurious correlations are expected to be interpreted by a model and its explanation.  

We will extend previous work on evaluating explanation methods' fidelity in the presence of Clever-Hans features by constructing a causal model. We extend a simple benchmark dataset with known ground truth to generate data using a structural causal model.
While the core or \textit{target} feature actually determines the class, other spurious features are still present and correlate with the it through a known indirect causal pathway.
Knowing the generating factors helps to quantify the ground-truth feature importance of not only the core and spurious but also irrelevant features.
With the aim of evaluating fidelity in the presence of a spuriously correlated feature in a relative way, a set of models with varying coupling between the core and the spurious feature are trained. The ground-truth importance of the spurious feature is calculated for each trained model. In expectation, the model's importance for the core feature declines and for the spurious feature increases as the coupling gets stronger.

If CRP indeed produces an accurate explanation, we expect that more concepts should assign higher relevance to the spurious feature the stronger its impact on the actual prediction of the model. It is important to note that the model might accurately predict based on the target feature even though the coupling ratio is high, when there are enough unaffected examples. Non-causally correlated features embody undesirable local minima in the cost function of a network which have to be overcome. Through refined learning procedures and due to the non-linearity of deep neural networks, they are remarkably robust to these artifacts. That means that they only use spurious features if either the random initialization was very unfortunate and made it impossible to escape from such a local minimum or if using the spurious feature is actually information-theoretically cheaper than the true target feature. 

If CRP were genuinely be faithful to a model, it would correctly follow its learned relationship, but it is also possible that CRP's attribution to the spurious feature will more closely follow the training data distribution or be otherwise disturbed. Our experiment therefore aims to illuminate and possibly answer the following question: 

\begin{quote}
\textit{Is the relative feature importance of a spurious feature explained by CRP in correspondence with the true importance assigned to it by the model, or is it more closely aligned to the associations within the data distribution?}
\end{quote} 

\filbreak
In summary we devise a strategy with the following steps to answer this question:
\begin{enumerate}
    \item We construct a causal model which has a spurious correlation using a benchmark dataset.
    \item We construct a causal explanation generating model which embeds this data model into the explanation context.
    \item By intervening on a variable of this explanation generation process which determines the coupling strength in the data distribution, we train a succession of neural network instances.
    \item In relationship to this coupling factor we establish a ground-truth of the models (spurious) feature importance.
    \item We construct metrics to measure the effect of this coupling of spurious and core feature on the concept-based approach of concept relevance propagation (CRP).
    \item Finally we compare the effects of the intervention on model importance and explanation importance
\end{enumerate}
\filbreak

\section{Outline}
To further motivate this approach we will summarize and put into relation previous work on XAI, evaluation of XAI, and the causality framework \cref{chapter:background}. We will lay down the formal definition of the XAI methods LRP and CRP and of the causal concepts applied here.
\Cref{chapter:method} introduces the causal explanation generation process, and how the benchmark dataset embeds into it. It also describes the methods used to establish a ground-truth \textit{feature importance} of the model in relation to the data distribution. 
The main part introduces metrics for measuring feature importance in concept-based explanations produced by CRP in \cref{section:measure}. The overall construction of the experiment and the set of trained models is summarized in \cref{section:experiment_setup}. Finally the different measures are visually and quantitatively analyzed in \cref{chapter:results} and discussed in \cref{chapter:discussion}.

\chapter{Theoretical Background}\label{chapter:background}

{ \color{red} 

about 20-30 pages (rather less I guess?)

\begin{enumerate}
    \item Introduction to XAI in general
    \item Evaluation of XAI methods in general
    \item Structural Causal Models and causal framework
\end{enumerate}
 }
 \todo{write background}

\section{Neural Networks}
\begin{itemize}
    \item Explain all general concepts that are needed for understanding CRP etc
    \item layers
    \item neurons
    \item convolutional blocks/ layers
    \item activation functions (especially ReLU)
    \item other types of layers? 
    \item backpropagation / forward
\end{itemize}

\section{Layerwise Relevance Propagation}
\begin{itemize}
    \item formula(s) for LRP
    \item explanation of LRP and briefly maybe Deep Taylor Decomposition
    \item overview of propagation rules \cite{Montavon2019}  \todo{understand all rules etc of LRP/CRP}
    \item current best practices for LRP and what is used for CRP
\end{itemize}

\section{Concept Relevance Propagation}
\begin{itemize}
    \item theoretical idea of Concept Relevance Propagation and what it seeks to improve
    \item Explain formula of conditional relevance in detail
    \item some examples of usage:
    \begin{itemize}
        \item relevance scores for \textit{concepts} (=neurons)
        \item relevance maximization images
        \item conditioning on single concepts/ neurons ...? 
        \item attribution graph 
    \end{itemize}
\end{itemize}


\section{Causal Framework}
\subsection{Structural Causal Models}

\begin{itemize}
    \item Explain and define in detail Structural Causal Models
    \item neural networks could be seen as SCMs \cite{Chattopadhyay2019}
    \item but AI / neural networks in general do not care about causation and work through finding useful correlations
    \item and that is good this way, otherwise they would never find anything useful, statistics and correlations are great
    \item none-the-less the better we get at identifying spurious features the more causal methods might apply? 
    \item it doesn't matter whether the network has found the actual causal reasons for its prediction, but explanations are a distinctively causal concept.
    \item and explanation asks how and why, so we want to know the cause of model predicting Y from X
    \item causal methods have started to be used for evaluation of xai
\end{itemize}

\subsection{Interpretation as Interventions}
???

\subsection{Data Generation Process}

Other?

\begin{itemize}
    \item Short introduction to causal effects
    \item counterfactuals
    \item 
\end{itemize}


\section{Evaluation of Explanations}
\subsection{Ground Truth Importance}
\begin{itemize}
    \item What are currently used ground truth importance measures for concepts or latent factors
    \item introduce Prediction Flip with formula or application to our use case 
    \item R2 score with formula \cite{Sixt2020}
    \item mean logit change with formula
    \item make clear: human understanding is the ultimate goal, so user studies are the gold standard (but often not well done) but not feasible here
    \item relate to constant vector shift problem and how this might be measured
\end{itemize}
\subsection{CRP Concept Importance Measures}
\todo{need proper measure}
\begin{itemize}
    \item explain the measures i use to score how well the concepts are separated
    \item show theoretical basis
\end{itemize}
\subsection{Causally somehow? }
\chapter{Problem Setting}\label{chapter:problem_setting}

\begin{figure}[t!]
    \advance\leftskip-2.3cm
    \includegraphics[width=1.4\textwidth]{thesis_latex_template/pics/pipeline_clean.png}%
    \caption[Pipeline]{Visual Summary of the steps of our experiment (best viewed digitally)}
    \label{fig:pipeline}
\end{figure}

The aim of this thesis is to compare the reaction of model and explanation towards spuriously correlated features using a causal framework.
We want to answer the question whether the causal effect of an intervention on a meta-variable in the causal data generation process on the explanation is exactly that of the model or whether the explanation reacts to a shift in the data distribution in other ways than the model. 
To this end, we devise a structural causal model seeking to mirror realistic causal pathways in image data to generate a toy dataset, which is detailed in \autoref{section:causal_model}. This is then embedded into a broader causal model describing how model predictions and explanations are generated.
We aspire to extract the measurable relative feature importance of the confounded but not causal feature for both the trained model and its explanation. In essence, this is the causal effect of intervening on this spurious feature in relation to its coupling with the target feature. 
Throughout this, we construct hypotheses of what makes relative feature importance \textit{measurable} within an explanation.

The authors of CRP claim that through looking inside of the model and computing conditional relevances of (sets of) hidden features, an explanation can not only answer \textit{where} but also \textit{what} questions \cite{Achtibat2022}. Hypothetically, this makes CRP more apt for explaining relative feature importance, which is why we focus on this method in our work.
In our first toy problem with a spatially separated spurious feature both of the questions are not so hard to answer. Instead we want to follow up with the question of \textit{how much} a feature influences the decision. Because CRP supposedly splits importance into multiple concepts, which it then assigns percentage-like relevance to, it could answer the \textit{how much} question, i.e. of relative feature importance, more intrinsically than general local attribution methods. The second toy problem mostly tests the constructed measures for a different scenario: Here, the spurious feature is spatially overlapping with the core feature and therefore the \textit{what} question returns to the center of attention. The \textit{how much} should however be just as measurable for such a case.

In distinction to other work, which looks at ground-truth importance mostly as a binary variable, we look at it in a relative and continuous way. This approach enables the hypothesis that a good explanation method's curve of relative importance with regards to the actual coupling within data should match the curve of the models relative use of the feature closely. It is motivated by the view that features in training data for realistic problems are never completely independent and instead a balance of the relative importance of features has to be decided. For that, it is also necessary to create a frame of reference. There is no inherent way to define the interval in which relative feature importance resides for one data instance, explanation method and dataset. By intervening on the causal data generation process we therefore hope to establish expected values of our measures with regards to the true feature importance.

The value of the meta variable, whose effect is measured, can be thought of as a coupling ratio between the truly important feature and a spuriously correlated feature in our experiments. We name this the \textit{measure} $m_0$ which is used for the comparison against the ground truth model importance measure $m_1$ and the explanation importance measures $m_2$. Measure $m_1$ is the ground truth of how strongly a model reacts to a changing latent factor and is constructed in \cref{section:gt_measure}. For $m_2$ we examine multiple possibilities of measuring relative feature importance for explanations produced by concept relevance propagation. The candidates are formally introduced in \autoref{section:measure}. 

To give justice to the potential for human intuition of the concept-based method we investigate, the goal is to also take into account how well relative feature importance is extractable for humans. Comparing the total numerical change of relevance per pixel when intervening, while it might be accurate, does arguably not represent the understandable effect of an intervention. We aim to include the complexity of what has to be extracted from the explanation into our measures. We thus test whether a reduced \textit{gist} of what humans could take from a concept-based explanation produced by CRP is similarly related to the models true relative feature importance as the whole explanation. 

Our hypothesis is on the one hand that if the similarity of $m_1$ and an instance of $m_2$ is high, we have potentially found a good measure of explanation fidelity and can also make claims to the explanation being faithful. On the other hand, if a measure accurately follows the true relative feature importance, this does not prove that humans can extract that information from the explanation well. We hence analyze the limitations of the constructed measures on the spatially separated spurious feature but especially on the overlapping spurious feature. 

In this approach lies the implicit assumption that one simple saliency map is not enough to identify relative feature importance of spurious features. The concept-based explanation might relieve some of the grievances with general local attribution but also increases complexity of an explanation. 
The question \textit{``which pixels positively or negatively influenced your decision?''} can be ill-posed, or might not give an answer well suited as the explanation to a models decision. For example, when an objects \textit{texture} is a Clever-Hans feature, the importance can be within the true boundary of the object to be learned, even though the wrong feature has been learned. There is evidence that even using concept-conditional explanations, like CRP does, is not fully enabling to identify and remove such non-localized or non-separable biases \cite{Dreyer2023a}. 

Our findings on small benchmark problems will have to be confirmed by applying the framework to varying generating causal models and interventions with other latent causal features in the future. 
It is also necessary to compare the results not only for the concept-based explanations generated by CRP but, where applicable, also for other explanation methods.


\chapter{Methods}\label{chapter:method}
still missing here:  short introduction to the chapter

\section{Causal Model}\label{section:causal_model}
The core aspect of our analysis is the combination of causal methods for data generation and ground-truth feature attribution. 
The main process includes a hyperparameter, such as the later discussed coupling ratio $\rho$, which one can intervene on, together with predictions and explanations generated for multiple trained instances of a model and its output. 
Within this greater structure a subprocess constructs the image dataset with a structural causal model, taking only $\rho$ as an input. While $\rho$ stays fixed for each model to be trained, it is a causal variable in the overarching structure. This way we create an array of causally generated datasets whose constitution only differs by this one variable. 

\subsection{Data Generating Causal Model}\label{section:data_gen_causal_model}
For the most exact comparison of attribution between a model and its explanation it is imperative to know the ground truth of the training data. A structural causal model (SCM) can define variables and the \textit{independent} mechanisms with which they interact precisely. We want to define an SCM that closely mirrors image generation processes as they happen in realistic scenarios. 
Explicitly using a generating SCM has previously been done to create or test new attribution methods \citep{Parafita2019, Wilming2023, Clark2023, Goyal2019, Reimers2019, Reimers2020}. A few other works evaluating XAI methods with a known ground truth have implicitly used structures akin to an SCM without defining it in causal language \citep{Kim2018, Yang2019, Arras2022}. 

The causal graph and its structural assignments used for our experiment are defined as follows: 
\begin{equation}
\label{eq:scm}
\begin{aligned}[c]
G&:=\eta^g \notag \\ 
N_w&:=\eta^w  \notag \\ 
N_s&:=\eta^s \notag \\ 
N_x&:=\eta^x  \notag \\ 
W&:=(\rho * G + (1-\rho)* N_w) \geq 0.5 \notag \\ 
S&:=(\rho * G + (1-\rho)* N_s) \geq 0.5 \notag \\ 
Z&:= \eta^z \\ 
\mathcal{X}&:= W + f_{image}(S, Z) + N_{x} \\
\end{aligned}
\begin{aligned}[c]
& \mathrm{with} \  \eta^g \sim \mathcal{N}(0.5,0.02) && \notag \\ 
& \mathrm{with} \  \eta^w \sim \mathcal{N}(0.5,0.1) && \notag \\ 
& \mathrm{with} \  \eta^s \sim \mathcal{N}(0.5,0.1) && \notag \\ 
& \mathrm{with} \  \eta^x \sim \mathcal{N}(0,0.05) && \notag \\ 
\\
\\
& \mathrm{with} \  \eta^z \sim \mathcal{U}(0,245760) && \notag \\ 
\\
\end{aligned}
\end{equation}

\begin{figure}[t!]
    \centering
    \includegraphics[width=0.9\textwidth]{pics/generating_scm.png}
    \caption[Data Generating SCM]{Data Generating Structural causal model.
        $\rho$ is not visible in the SCM. It determines the signal-to-noise ratio between the confounding $G$ and the noise variables $N_w$ and $N_s$ in the structural assignments. The noise terms of $Z$ and $G$ are not shown explicitly.}
    \label{fig:generating_scm}
\end{figure}

The SCM, as seen in \cref{fig:generating_scm} and \cref{eq:scm}, serves as a ground truth for our experiment. It is mostly an additive model using Gaussian distributions for the noise terms, except for $Z$ which uniformly at random selects other image generating factors like rotation, scale and position of the shape. Also, the function $f_{image}$ generating the images out of the latent factors and the shape is not further specified.

The meta-variable $\rho$ adjusts how much information is shared between the true class information (shape), which we name \textit{core feature} following \cite{Singla2022} and the watermark or \textit{spurious feature} through a shared common ancestor named \textit{generator} $G$. For each model, coupling ratio $\rho$ is fixed to later enable a comparison between models trained with different coupling ratios. 
A second parameter which we keep fixed for all experiments, determines how frequent the spurious feature is in the data. We set this parameter which can be seen as the \textit{prevalence} to one half, so that just as many images have a watermark as not. Notably, when $\rho$ is one this means that all ellipses will have the watermark and no rectangle will and when $\rho$ is zero there is no correlation of watermark and shape. 

It is important to note that this particular SCM is just one of many possible ways to model how spurious features might interact with core features. It attempts to follow the logic of how images are generated or selected in real datasets. Here, it particularly looks at pathways for the generation of Clever-Hans or watermark features, often present in image datasets \citep{Lapuschkin2019}. As visible in \cref{fig:equivalent_scm} different causal models can produce an equivalent distribution of the two latent factors in question (watermark and shape). One can think of more variations of SCMs which are able to produce the same correlation, so the choice of using the confounder version as in \cref{fig:generating_scm} is mainly due to its ease of implementation. It also lends itself best to the Clever-Hans task, because in this SCM there is no real causal effect between the core and spurious feature. It would therefore be ideal for the a neural network to ignore the spurious feature completely. Without assumptions about the generating mechanism though, the confounder case is not identifiable, or in other words, distinguishable from $W \rightarrow G \rightarrow S$ and $S \rightarrow G \rightarrow W$ because it is Markov equivalent to those scenarios. 
Although the presented collider case (\cref{fig:equivalent_scm}\textbf{a}) is theoretically distinguishable from the confounder case (\textbf{b}), one can not hope that a network which only has access to the intervened on (selected) data will recover this. Neither can we hence expect the explanation to make such a distinction.

The idea of investigating the effect of a coupling ratio $\rho$ was inspired by \citet{Clark2023}, who also used a \textit{signal-to-noise} ratio in their generating model. Instead of a confounding model their learned data instances are colliders of the label and a suppressor variable similar to \cref{fig:equivalent_scm} \textbf{a}. In their experiments the expected explanation importance of a suppressor feature ought to be zero, as long as the collider is not intervened on, but also when the suppressor and core feature are coupled through smoothing of the image. We contest this assumption, as it is not clear whether a neural network does not implicitly condition on this collider (which is the image set), therefore introducing correlation between the parent components through a different kind of selection bias. We disagree with their statement that a good explanation method ought to not attribute any importance to such suppressor variables. This is, as  \citet{Nauta2023} put it, a case of conflating external \textit{coherence}, i.e., ``agreement with human rationales'' \citep{Atanasova2020} with \textit{correctness} towards the model. As we are not aware of any explicit way in which deep neural networks differentiate between a suppressor or confounder variable, it is unwarranted to expect an explanation to explain something the model does not even conceptualize. 

The direction of causal links for image classification is, as can be seen from this example, highly debatable and shall not be the focus of this work. Whether the selection of the AI researcher reducing costs with free images resulted in classes being polluted with watermarks (\cref{fig:equivalent_scm}\textbf{a}), or whether a scientist marked x-rays with their diagnosis (\cref{fig:equivalent_scm}\textbf{b}) seems to not be distinguishable for ML models. 
Instead, we only want to find to what degree a neural network learns and attribution method explains the distribution generated by a particular SCM.

This first experiment is testing whether the framework we apply is in principle reasonable and to establish a baseline for how feature importance behaves. To further illuminate its applicability one needs to look at other SCMs with different datasets. We only compare the result of this experiment to a scenario where the core feature to one where they overlap but the SCM stays the same. 
It would of course also be interesting to see if our framework also applies to the suppressor variable model or other SCMs.

\begin{figure}[t!]
    \centering
    \includegraphics[width=0.8\textwidth]{pics/equivalent_scm.png}
    \caption[Selection vs. Confounder Bias]{SCMs typically found in image datasets.
    \textbf{a.} selection bias \textit{(researcher chooses images from free online collection with watermarks)}
    \textbf{b.} confounder bias \textit{(e.g. scientist marks positive x-ray scans with sign)}}
    \label{fig:equivalent_scm}
\end{figure}

\subsection{Explanation Generating Model}
The data generation causal model is part of the model which generates predictions and explanations.
This model is defined in a similar way to the \textit{explanation generating process (EGP)} by \citet{Karimi2023}.
Ratio $\rho$ is a meta-variable of our image generation process in a similar sense to how hyperparameters are defined for the training there. While \cref{fig:generating_scm} depicts the data generating causal model (DGCM) of the training dataset in more detail, \cref{fig:egp} shows how this is embedded into the mechanism of generating explanations. 

\begin{figure}[t!]
    \centering
    \tikzset{%
        neuron/.style={
            ellipse,
            draw,
            minimum height=8mm,
            },
        arrows={[scale=1.2]}
    }
%\begin{mdframed}[backgroundcolor=graybg]
    \begin{tikzpicture}[]
    % generating model:
        \node [neuron]  (r) at (0,4) {$\rho$};
        \node [neuron]  (rs) at (5,3) {seed};
        \node [neuron]  (scm) at (3,4) {DATA SCM};
        \node [neuron]  (w) at (7,4) {weights};
        \node [neuron]  (x) at (7,2) {$\mathcal{X}$};
        \node [neuron]  (p) at (9,3) {$Y$};
        \node [neuron]  (e) at (11,2) {$E$};
        
        \draw[->] (r) -- (scm);
        \draw[->] (scm) -- (w);
        \draw[->] (rs) -- (w);
        \draw[->] (w) -- (p);
        \draw[->] (x) -- (p);
        \draw[->, bend left=40] (w) to (e);
        \draw[->, bend right=20] (x) to (e);
        \draw[->] (p) -- (e);
    \end{tikzpicture}
%\end{mdframed}
    \caption[Explanation Generation Process (EGP)]{Explanation Generation Process \textit{EGP}}
    \label{fig:egp}
\end{figure}

\subsection{Watermark Benchmark Dataset W-dSprites}\label{section:dataset_wdsprites}
Although this thesis is not the first work to use a toy dataset with known generating factors to evaluate attribution methods, we newly adapt a dataset which is as simple as possible and yet mirrors the main workings of a realistic computer vision problem. For this, we alter the dSprites dataset \citep{dsprites17} by adding small watermarks in the shape of \textit{w}s to some images. The dSprites dataset was originally constructed as a means for testing the degree of disentanglement a machine learning model has achieved. It contains circa 700k images with rectangles, ellipses or hearts in varying positions, scales and rotations. To simplify the task more we only use the rectangle and ellipse class for our experiment. Another motive is that in a binary classification task positive and negative relevance might be used in varying strategies for prediction. In theory this could make the class-insensitivity studied by \citet{Sixt2020} visible or counteract it. Further details on our dataset can be found in \cref{appendix:dsprites}.

\begin{figure}[t!]
    \centering
    \includegraphics[width=0.6\textwidth]{thesis_latex_template/pics/examples_datasets.png}
    \caption[Example Images W-dSprites Watermark Scenario]{First row: images from the original dSprites dataset. 
    second row: images from the watermark-dSprites dataset with small \textit{w} as a watermark on some images in random positions at the edge of the image and Gaussian background noise added. 
    third row: images from the overlap-dSprites dataset with either sharp or blurred noisy patterns and Gaussian background noise added.}
    \label{fig:dsprites_examples}
\end{figure}

\subsection{Overlapping Scenario Dataset}\label{section:dataset_overlap}
In addition to the first benchmark dataset, we deemed it necessary to conduct our experiments on a second type of problem.
The case where a spurious feature is spatially separated from the target feature has been addressed quite often and successfully with local attribution methods. Yet, in the case where features are overlapping it is not even clear how an attribution map should react ideally and how to measure the feature importance for spurious features. To analyze this case, we have created another dataset in which the pattern inside of the shape is biased. The data-generating as well as the explanation-generating model and all other parameters stay exactly the same as in the first experiment.
Images, where the spurious feature $W$, which was the watermark in the first scenario, takes the value $W=0$, have a Gaussian noise as the pattern of the shape. If $W= 1$, instead, a Gaussian noise with slightly higher variance is blurred by a Gaussian filter. This produces images in which the shape appears either noisy or blurry as seen in row three of \cref{fig:dsprites_examples}. 

The authors of CRP claim that their \textit{glocal} method enables humans to also identify non-spatial features like color or texture. Our idea is further motivated by computer vision problems where certain objects appear blurry more often because they often move fast or where bad image quality is biased.
To not necessitate adding color channels to the architecture and change the setup of our experiment and also because the biasing of color produces another kind of \textit{spatial separation}, we chose to use monochrome patterns.
For better intuition we will mostly reference the first case, where the spurious feature is a separable watermark, when laying out the next steps. Nonetheless, the measures are also applied in the same way to the pattern scenario. 

\section{Generating Explanations with Concept Relevance Propagation}\label{section:explanations_with_crp}
The previously described causal framework can be applied to a multitude of explanation methods which have concept-specific attribution maps. However, we limit our analysis to CRP and interpretation techniques constructed with CRP. This is mostly due to the time limitations of a master thesis, but also because its authors specifically claim in their paper that CRP facilitates identifying relative importance. 
Producing explanations using CRP requires decisions on the backpropagation rules, on the conditioning sets and further hyperparameters. 
We follow the recommendations and default settings of CRP's authors \citep{Achtibat2022, Achtibat2023} and best practices \citep{Kohlbrenner2020} as closely as possible.
For the backpropagation we apply the $LRP_{\varepsilon -z^+- b^-}$ - rule as recommended by \cite{Kohlbrenner2020}. Due to the simplicity of our CNN model no more model canonization steps need to be applied. See \cref{appendix:lrprules} for further technical details. 

\begin{figure}[t!]
    \centering
    \includegraphics[width=0.8\textwidth]{thesis_latex_template/pics/conditional_heatmaps.png}
    \caption[Comparing Attribution Maps of Layers]{Concept-conditional heatmaps for one example image. Note the Sobel-filter-like attributions in earlier layers and the more combined attributions of watermark and shape in later layers.}
    \label{fig:cc_heatmaps}
\end{figure}

Principally, neurons in every layer of a model can be conditioned on using the CRP approach. However, the resulting attribution maps are not necessarily depicting disentangled \textit{and} abstract enough concepts. When looking at concept-conditional attribution maps from the earlier layers, one will likely see low-level features akin to edge detection filters. In the late, fully connected, layers before the output the previously disentangled concepts might get mixed together again for the final decision. In \cref{fig:cc_heatmaps} an example shows the tendency from trivial to abstract concepts. 
According to \cite{Dreyer2023a}, who refer to \cite{Zeiler2013}, the \textit{last convolutional layer} is ``most likely representing disentangled representation''. This is why we use the third and last convolutional layer of our minimal model for the analysis. It is not clear whether the extremely small size of our model hinders a transfer of the results to realistic scenarios. Yet, training significantly larger models would have been too computationally expensive for this principled approach, requiring many trained models.
In the following we will reiterate the steps necessary to produce the different components of CRP-explanations in our experiment.

\subsubsection{Concept-Conditional Attribution Maps and Relevance for Prediction}
The concept-conditional backpropagation rule described in \cref{section:crp_background} can be applied to arbitrary sets of neurons $\theta$. In our scenario we create class-specific attribution maps conditioned on each individual neuron (termed concept $c$) in the selected layer. For this, we use the output as the initialization for relevance, keep all other layers untouched and then mask out the desired neuron's relevance in the layer $\ell$: 
\begin{equation}
    R^{\ell}_{c}(\mathrm{x} |\theta_{c}) = \sum_{i} R_i^{\ell}(\mathrm{x} |y \cup \theta_{\ell} = \{c\})
\end{equation}
Here $i$ represent all relevances in lower layers that are a part of the concept $c$ and not masked out. 
To yield the attribution map, the importance is back-propagated through all layers until the input layer $\ell = 1$ and not summed per input feature $i$, producing individual relevances termed $R_{i}^{1}(\mathrm{x} |\theta_{c})$. In the following we will refer to this concept-conditional attribution map as $\mathcal{A}_c(\mathrm{x})$ and the class specific relevance of concept $c$ as $R_c^{\ell}(\mathrm{x})$. Due to the \textit{conservation laws} that CRP inherits from LRP (compare Equation 7, \cite{Achtibat2022}) the relevances $R_c^{\ell}$ within one layer can be interpreted as the \textit{percentage} of importance going through concept $c$. To enable this view also for out-of-distribution samples the authors recommend normalizing the relevances to sum to 1:
\begin{align}\label{eq:normed_relevance}
    R^{\ell}_{c,norm}(\mathrm{x} |\theta_{c}) = \frac{R^{\ell}(\mathrm{x} |\theta_{c}) }{\sum_k |R_k^{\ell}(\mathrm{x} |\theta_{c})|}.
\end{align}

\subsubsection{Local Concept Importance}
To find out how much a certain region contributes to the overall prediction but also which concepts are most highly activated within that region, the authors propose local concept importance. 
This method simply masks out the desired region (or input pixels), within the concept-conditional attribution map and sums the relevance within that mask. 
Again, for readability we from now on refer to this local concept importance in a region $B$ as $R_B$.

\begin{align}\label{eq:local_importance}
    R_{B}^{\ell}(\mathrm{x} | \theta_c) = \sum_{(p,q) \in B} R_{p,q}^{\ell}(\mathrm{x} | \theta_c)
\end{align}

\subsubsection{Relevance Maximization}
As described in the background \cref{section:crp_background}, CRP's authors also use concept-conditional relevance to create prototypical reference sets of each concept. In a human subject study \citep{Achtibat2023} they find this method to be useful for the identification of Clever-Hans artifacts, where only using heatmaps fails. We therefore aim to include this technique into our analysis. The reference sets consist of  images with maximal relevance for a neuron. Each image can be further concentrated on the encoded concept by thresholding the relevance and by cropping to the receptive field of the neuron in question. This yields a set of cropped images which ought to describe the encoded concept. 

As mentioned by \citet{Achtibat2022}, using the whole dataset might make the selected images too similar to each other. Hence we compute the maximization for a subset of only 300 images to enforce more variance. This is especially necessary for the dSprites dataset because the difference between images is small. Importantly, the sample we select from does not have an association between $S$ and $W$, meaning that watermarks (or blurry patterns) are equally likely to occur on rectangle as on ellipse images. 
The provided code produces sets of 40 images for each neuron in each layer with either relevance or activation maximized using the sum or maximum (conditional) relevance/activation for an image. We mainly use the default option of maximizing for the sum of relevance of an image for a concept, but also look at the results when the maximization target is the maximal relevance of an image or when activation is used instead of relevance. 
The implementation of this and all previously described methods is made accessible by CRP's authors at
\url{https://github.com/rachtibat/zennit-crp}. 

Relevance maximization reference sets are denoted the following way:
 \begin{align}
& \mathcal{T}_{sum}^{rel} (\mathrm{x}) = \sum_{c} A_c(\mathrm{x}|\theta) \\
& \mathcal{T}_{max}^{rel} (\mathrm{x}) = \max_{c} A_c(\mathrm{x}|\theta) \\
&\mathcal{X}^{\star} = \{ \mathrm{x}_{1}^{\star},\mathrm{x}_{2}^{\star},..., \mathrm{x}_{n}^{\star} \} = arg \sort_{\mathrm{x} \in \mathcal{X}} \mathcal{T}(\mathrm{x}) \\
& \mathcal{X}_{k}^{\star} = \{ \mathrm{x}_{1}^{\star},\mathrm{x}_{2}^{\star},..., \mathrm{x}_{k}^{\star} \} \subseteq \mathcal{X}^{\star}
 \end{align}
\section{Data Ground Truth Correlation $m_0$}
The goal of this analysis it to gather information on how a known coupling ratio of two features interacts with their importance to the model and their explained importance. 
Measure $m_0 = \rho$ is the correlation between the shape and spurious feature in our data generating model. When $\rho$ is zero, the features are not associated at all, when it is one they correlate maximally. Conceived as a \textit{signal-to-noise} ratio between the correlated and uncorrelated parts of $S$ and $W$, it can directly be used as a measure of the coupling of spurious (watermark) and core (shape) feature in the data distribution. However, the data generating SCM introduces a small modification due to the binarization of the variables $W$ and $S$. It might therefore be more accurate to look at the actual correlation of the binary features in the generated data distribution as a ground-truth (see \cref{fig:finding_rho}). Considering that we have two binary variables, their correlation can be measured using the $\phi$-coefficient. It is also called \textit{Matthews} or \textit{Yule-phi} coefficient and can be interpreted as the Pearson correlation coefficient for two binary variables. For our symmetric model, where the number of ellipses is always equal to the number of rectangles and always exactly half of the instances share their value for the spurious feature, the formula is further simplified:

\vspace{1em}
\begin{minipage}[t]{0.45\textwidth}
\begin{tabular}{|c|c|c|c|}
    \hline
     & $W= 1$ & $W = 0$ & total  \\  \hline
    $S= 1$ & $n_{11}$ & $n_{10}$ & $n_{1*}$ \\ \hline
    $S= 0$ & $n_{01}$ & $n_{00}$ & $n_{0*}$ \\ \hline
    total& $n_{*1}$ & $n_{*0}$ & $n$ \\ \hline
\end{tabular}
\end{minipage}%
\begin{minipage}[c]{0.53\textwidth}
\begin{align}
\phi & = \frac{n_{11} * n_{00} - n_{10}*n_{01}}{\sqrt{n_{1*}*n_{0*}*n_{*0}*n_{*1}}} \notag \\
n_{11} = n_{00}, &n_{01} = n_{10}, n_{1*} = n_{0*} = n_{*1} = n_{*0}   \notag \\
\rightarrow |\phi| & = 1- \frac{2 \cdot (n_{10} + n_{01} )}{n} \label{eq:simplified_phi}
\end{align}
\end{minipage}
\vspace{1em}

Generally, we do not expect the model to perfectly reconstruct the coupling ratio $\rho$ or the binarized correlation $\phi$. After all, the strength of neural networks presumably lies in recovering the truly important feature even when other, highly correlated features are present. However, some research expects explanations to give insight into the distributions of the training data to better understand how biases might occur, even if a model has apparently learned to ignore spurious features \citep{Kindermans2017}. 

\begin{figure}[t!]
    \centering
    \includegraphics[width=0.6\textwidth]{thesis_latex_template/pics/gt_m0_phi_only.png}
    \caption[True Data Distribution $m_0$]{$\phi$-coefficient between $W$ and $S$ of sampled training data distributions with growing coupling ratio $\rho$}
    \label{fig:finding_rho}
\end{figure}

\section{Model Ground-Truth Feature Importance $m_1$}\label{section:gt_measure}
After having defined a ground truth for the coupling of $S$ and $W$ in the data ($m_0$), we now propose a few metrics yielding $m_1$, which is the feature importance of the spurious feature $W$ according to a model.
In contrast to realistic application scenarios our causal framework and artificially generated dataset enables us to establish the ground truth importance of features for a trained model. Similar to recent work on causal attribution \citep{Goyal2019,Parafita2019,Karimi2023} the causal effect of an intervention upstream on the output of a model can be estimated in the following way:
\begin{center}
Average Causal Effect of latent factor $W$ on output $Y$ \\
\begin{equation}
\displaystyle ACE = \mathbb{E} [ Y \ | \ do(W=1) ] - \mathbb{E} [ Y \ | \ do(W=0) ] 
\end{equation}
\end{center}

The intervention on a given latent factor is equivalent to this formula here, as our factors of interest are binary variables (watermark and shape). Due to our knowledge of the ground truth we can fix all other independent latent factors and feed one image with the watermark and the same image without it through the neural network, thereby achieving a pure intervention on $W$.  
However, it is not as clear how to define the output of a neural network. One can either measure the average causal effect on the binary prediction or on the output layers' logits, which change in a continuous fashion.
To account for the effect of the initialization of model weights and biases on the usage of either feature, we average each measure over multiple random seeds.

\subsection{Prediction Flip}
Computing the \textit{Prediction Flip}, as \citeauthor{Sixt2022a} call it in their experiments, is straight forward in our example as we can control all variables. 
This binary causal effect can be estimated as the percentage of images for which the prediction changes, when the factor $W$ is changed, which in turn is equivalent to the magnitude of the $\phi$-coefficient between prediction $Y$ and spurious feature $W$ in our scenario. Therefore, this measure is apt for the comparison with the $(S,W)$-correlation in the training data distribution.

For better readability we will refer to an image with the watermark $\mathrm{x}_{do(W=1)}$ as $\mathrm{x}$ and the same image without the watermark $\mathrm{x}_{do(W=0)}$ as $\mathrm{x'}$ in the rest of the thesis. As we average most metrics over a sample set of 128 images, selected uniformly from the full dataset, we denote this set as $\mathcal{X}$ from here on. The Prediction Flip (PF) for one model is then:
\begin{align}
\displaystyle 
& PF =\tfrac{1}{|\mathcal{X}|} \sum_{\mathrm{x} \in \mathcal{X}} |y(\mathrm{x}) - y(\mathrm{x'}) |. 
\end{align}

\subsection{Mean Logit Change}
For the W-dSprites classification task, the output vector consists of 2 logits $y_0$ and $y_1$. The model predicts \textit{rectangle} when $y_0 > y_1$ and \textit{ellipse} otherwise. To compute the mean logit change when intervening on our spurious feature $W$, we take a sufficient amount of samples $\mathcal{X}$ from our dataset and feed them through the model. First we predict for images with $W=1$ (containing a watermark) then for the same images with $W=0$. 
To enable better comparison we apply the soft-max function to the outputs to yield confidences, as during the training process. This keeps the relative magnitudes within the sample set intact but brings them to the range $[0,1]$. 
We expect this variant of the model importance to be slightly more sensitive to the spurious watermark feature $W$ for lower values of $\rho$ than the prediction flip. The reason being, that while the continuous output vector, i.e., confidence, might already be affected by the spurious feature for weakly biased models, the prediction will only change once the spurious feature becomes more easy to identify than the core feature. Vice versa, the confidence when only the spurious feature is used for prediction could be low and hence the prediction flip higher than the mean logit change.

It is important to choose a reasonable distance measure between the two output vectors, yet there is no agreed upon distance metric for the assessment of similarity of attribution maps and vectors. \citeauthor{Sixt2022a} and \citeauthor{Goyal2019} use the mean absolute difference (or $L1$-norm). 
Recently, assessing the similarity of activation and relevance vectors has often been done using the cosine similarity \citep{Sixt2020,Achtibat2023,Dreyer2023a, Pahde2023}. Therefore, we include the cosine distance as a potential distance metric for the model output to enable comparison to the explanation. \citeauthor{Karimi2023} test different kernels for a kernelized treatment effect on multidimensional outputs and find that the relative effects are not sensitive to the choice of kernels. 
We therefore assume that the trend of effects will not be majorly affected by the choice of a distance metric but still test this hypothesis.


\begin{align}\displaystyle 
& \text{Mean Absolute Distance:} & \notag \\
&\MLC_{\rho, m}^{abs} = \tfrac{1}{|\mathcal{X}| * 2} \sum_{\mathrm{x} \in \mathcal{X}} 
|y_0(\mathrm{x}) -y_0(\mathrm{x'})| + |y_1(\mathrm{x}) -y_1(\mathrm{x'})| \\
& \text{Mean Squared Distance:} & \notag \\
& \MLC_{\rho, m}^{sqa} = \MLC_{\rho, m}^{abs}^2\\
& \text{Cosine Distance:} &  \notag \\
& \MLC_{\rho, m}^{cosine} = \tfrac{1}{|\mathcal{X}|}\sum_{\mathrm{x} \in \mathcal{X}}  
1 - \frac{\vec{y}(\mathrm{x}) \cdot \vec{y}(\mathrm{x'})}
{\lVert \vec{y}(\mathrm{x}) \rVert \lVert \vec{y}(\mathrm{x'})\rVert }
\end{align}



\section{CRP Explanation Importance $m_2$}\label{section:measure}
Our goal is to compare the causal effect of an upstream intervention on the true feature importance $m_1$ and then the explained feature importance $m_2$. After having defined ways to measure the models true feature importance, we need to repeat the process for the concept-based explanation produced by CRP. It is not well known how humans perceive changes in an explanation and there is no agreed upon scale of importance, so we believe it is best to construct multiple measures to test against each other. Some of the proposed metrics are derived from existing work on evaluating feature importance for local attribution methods \citep{Sixt2020, Karimi2023, Arras2022}. Each candidate should be a variation of measuring the average causal effect of intervening on $W$ on an explanation $e$:
\begin{center}
\begin{equation}
\displaystyle ACE = \mathbb{E} [e \ | \ do(W=1) ] - \mathbb{E} [ e \ | \ do(W=0) ]
\end{equation}
\end{center}

The core question is, whether this effect over different $\rho$ is comparable between the explanation and the prediction. A perfect explanation assigns just as much importance to a feature as the model. This is one way to evaluate the fidelity of the explanation to the model. Some of the proposed measures however attempt to incorporate other desirable characteristics of XAI such as \textit{compactness} \citep{Nauta2023}. This also has the practical reason that while an attribution map is visually compelling, it is not a very concise description of relevance. Measuring the causal effect purely on the saliency maps also has the disadvantage that effects invisible or non-distinguishable by humans are accounted for. 
We thus aim to estimate the explanation importance not only through the attribution maps but also in less complex forms. 

The measures introduced in the following are roughly ordered from measures being most true to the numerical effect of intervention on the explanation to measures reducing the complexity of the explanation. Although human perception is not part of our evaluation, aiming for less complex explanations seems to be more in line with a \textit{good} explanation. While the first measure would require a pixel-wise comparison of multiple heatmaps to find differences, later metrics attempt to work with reduced and hopefully more human-interpretable abstractions. 

Related work often removes negative relevances in attribution maps to enable comparison with XAI methods that do not measure negative relevance. Another reason could be that incautious aggregation of attribution results with negative and positive relevance introduces cancelling-out effects. We believe, however, that a spurious feature can be negatively as well as positively attributed for a model to be biased. Therefore, the measures equally incorporate both the magnitude of the positive and negative relevance. In our scenario with a binary classification task it is reasonable to assume that whatever is positively attributed for one class, should be negatively attributed to the other with the same magnitude, but we suppose that this should be tested by incorporating both.

\subsection{Mean Attribution Change}\label{section:measure_mac}
The likely most straight-forward way to calculate explanation importance for neurons in a layer using CRP is to measure the average causal effect of an intervention on the concept-wise attribution directly. This approach aims to emulate the \textit{mean logit change} of the prediction for the concept-based explanation. In a given layer $\ell \in L$, the relevance of each of the neurons $c \in \ell$, which we described as $R_c^{\ell}(\mathrm{x})$ in \cref{eq:normed_relevance} and the pixel-wise attribution maps $A_c(\mathrm{x})$ can be used for this.
The causal effect of intervening on the spurious feature on one neuron is then the difference between its attribution map $A_c$ or relevance $R_c^{\ell}$ of one image where $W=1$ and the same image with $W=0$. If the model has indeed learned separate concepts for each feature, the same effect should become visible both for the attribution maps and the aggregated relevances.

To look at the difference in explanation, we compare multiple types of dissimilarity between two sets of heatmaps (i.e. 2-dimensional pixel maps). Firstly, we look at a normalized absolute pixel-wise difference, or absolute error ($AE$). The mean squared error $SE$ also seems a natural approach and has been applied by \citet{Karimi2023} previously. Because we compute the $cosine$ distance for the relevance values $R_c^{\ell}$ as done among others by the authors of CRP \citep{Achtibat2023}, we also compute this metric for the high-dimensional attribution maps for completeness. This metric is especially applicable because it focuses on non-empty indices and normalizes the results independently of the dimensionality of the input. Note that we scale this metric by one half, because when negative and positive values are present, its maximal value is two.

A sensible normalization for a set of attribution maps, which is needed for the absolute and squared distance metrics, is not as trivial as for the output vector or the relevance vector. The naive approach would be, to take the maximally opposing images' difference. But it is not to be expected that any method would assign positive or negative values of full magnitude to every pixel. Instead, the results of attribution methods only sparsely assign attribution to few pixels, usually within regions of objects and not the background. 
We therefore have to find a smaller divisor which scales a maximal difference between a set of attribution maps with and without watermark to approximately 1.
To adjust for the sparsity of the attributions, we take the maximal sum of absolute values of all heatmaps over all samples for one model. This \textit{maximal absolute sum} scaling is very similar to what \citet{Achtibat2022} propose for the normalization of relevances of a layer, yet applied to more dimensions and multiple samples.

Here, we show the described distance metrics for the attribution map matrices. $A(\mathrm{x})$ represents an array of all $|\ell|$ (here 8) attribution maps of the concepts $c$ in a layer. The same metrics are computed for the relevance vector, which in our case consists of $|\ell|$ normalized relevance values $R_{c}^{\ell,(norm)}$. For these, the computation is considerably simplified as no aggregation per pixel is necessary. In principle, each of these distance metrics should produce comparable results. However, like \citet{Karimi2023}, we want to ensure that the choice of a distance metric has no adverse interaction with our results and therefore compare multiple candidates.

\begin{align}
\displaystyle 
& A_{tot}^{\ell}(\mathrm{x}) = \sum_{c \in \ell} \sum_{(p,q) \in \mathrm{x}} |A_{p,q,c}(\mathrm{x})|  \label{eq:total_absolute_relevance} & \\
& \max_{\rho, m}^E(\mathrm{x}) = \max_{c \in \ell} (A_{tot}^{\ell}(\mathrm{x}) , \  A_{tot}^{\ell}(\mathrm{x'}) ) & \\
& \MAC_{\rho, m, \mathrm{x}}^{AE} = 
\sum_{c \in \ell} \sum_{(p,q) \in \mathrm{x}}| \frac{A_{p,q,c}(\mathrm{x})}{\max_{\rho, m}^E} -\frac{A_{p,q,c}(\mathrm{x'})}{\max_{\rho, m}^E}| & \\
& \MAC_{\rho, m, \mathrm{x}}^{SE} = 
\sum_{c \in \ell} \sum_{(p,q) \in \mathrm{x}} \left( \frac{A_{p,q,c}(\mathrm{x})}{\max_{\rho, m}^E} -\frac{A_{p,q,c}(\mathrm{x'})}{\max_{\rho, m}^E}\right) ^2 & \\
& \MAC_{\rho, m,\mathrm{x}}^{cosine} = \left( 1- 
\frac{A(\mathrm{x}) \cdot A(\mathrm{x'}) }{|A(\mathrm{x})|\cdot |A(\mathrm{x'})|}\right) * \tfrac{1}{2} &
\end{align}

For each distance metric we aggregate results over a set of sample images for all models using the same value of $\rho$:

\begin{align}\label{eq:ace_metric}
& ACE_{metric} = \frac{1}{|M_\rho|\cdot |\mathcal{X}| }\sum_{m}^{M_{\rho}} \sum_{\mathrm{x,x'}}^{\mathcal{X}} Metric(\mathrm{x,x'})
\end{align}

We find it necessary to make the following distinction: A filter's relevance can change strongly when intervening on the watermark. But if the importance of the watermark is not perceivable from the associated conditional attribution map, it is not possible to decipher which concept that filter encodes without access to ground-truth concepts. Reasons why that could happen are, for example, scaling differences between two heatmaps or attribution in areas of the image that humans do not perceive as regions of interest (like the area of the watermark).
So although these metrics are constructed to measure the causal effect of intervention on the explanation as closely as possible, they down-play one of the core ideas of local attribution methods' potential usefulness. If adding or removing a watermark has a significant effect on the attribution to some pixels far away from it, this is not in line with desirable properties of a good explanation such as interpretability or even fidelity to the ground truth importance. It is likely that through the coupling of the spurious feature in our experiment, the changes of (especially pixel-wise) relevance do not only affect the watermark region itself but at least reduce the shape's importance too. 
Another potential downfall for human intuition is the way more localized concepts are understood in comparison to more global or wide-spread concepts which \citet{Achtibat2022} point out in their work. We therefore want to harness the potential of local concept importance for the application scenario with a spatially separated watermark.
In the \cref{section:region_specific} we thus introduce measures which concentrate on the attributed relevance within the ground-truth region and relative to the rest of the attribution map. This, to a degree, also reduces the complexity of what has to be interpreted from the explanation.

\subsection{Importance in Ground-Truth Region}\label{section:region_specific}
Arras et al. \cite{Arras2022}, apply two metrics for the analysis of importance in pixel maps. Relevance Mass Accuracy (RMA) measures the ratio of relevance within the boundary of a feature to the total relevance. Relevance Rank Accuracy (RRA) rates the percentage of pixels in such a bounding box that fall within the $k$ most important pixels in the heatmap. If those metrics are successful in describing the relative importance of a feature for the explanation too, which \cite{Yang2019} have already investigated, they should be applicable in our experiment. 

\subsubsection{Relevance Mass Accuracy (RMA)}
When trying to establish whether a feature is important, one would intuitively look at the feature itself first. Consequently, approaches like Relevance Mass Accuracy take into account the known boundary of a feature for benchmarks that have available information about the location and shape of features.
Bau et al. \cite{Bau2017, Bau2020} define a related measure to compare explanation importance to a ground truth using IoU (\textit{Intersection over Union}). These works assume the perfect attribution of importance to one feature to be a binary mask, which is one inside and zero outside the feature's region.
As noted in \cite{Arras2022} it is not known whether this perfect score is attainable or even desirable for an explanation. Yet the overall tendency should become clear. It is still more unclear, how this score should play out in a setting where the feature in question is only spuriously correlated to the target feature. Yang et al. \cite{Yang2019} formulate a model contrast score, comparing models that have learned differing importance for such a feature, to yield a relative feature importance. While our analysis goes in a similar direction, the ultimate goal would be to relate the feature importance of a spurious feature directly with the core feature without training multiple models.

First, we however apply strategies resembling the evaluation scores RMA, RRA and Pointing Game. 
Fortunately, the authors of CRP have already formulated \textit{local concept importance} $R_B$, evaluating the conditional attribution to a concept in a predefined region by masking the rest of the attribution out. Relating each neuron's local importance to its total importance cancels out the magnitude between the neurons. Therefore we scale the relative feature importance of one neuron by the total absolute relevance of all neurons $A_{tot}^{\ell}$ (see \cref{eq:total_absolute_relevance}).

In our case, the region of interest are pixels in the bounding box area $B$ around the watermark (see \cref{fig:bounding_box}). Note that in contrast to some works using the exact boundary of the object in question, we mask a generous margin around the watermark itself. The reason being, that the model has a max-pooling layer through which importance gets smoothed out around the actual pixel. 

\begin{figure}[t!]
\begin{minipage}[t]{0.45\textwidth}
    \vspace{-\topskip}
        \includegraphics[width=\textwidth]{thesis_latex_template/pics/bounding_box.png}
\end{minipage}
\begin{minipage}[t]{0.45\textwidth}
\begin{align}\displaystyle
& R_B = \mathrm{ see \ \cref{eq:local_importance}} \\
& A_{tot}^{\ell} = \mathrm{ see \ \cref{eq:total_absolute_relevance}} \\
& RMA(\mathrm{x}) = \sum_{c \in \ell}  \frac{
R_{B,c}(\mathrm{x})}{A_{tot}^{\ell}(\mathrm{x})|}
\end{align}
\end{minipage}
\caption{Boundary $B$ around watermark}
\label{fig:bounding_box}
\end{figure}

In our first experiment the shape and watermark feature are spatially separated, so local concept importance is a feasible instrument. This region-specific relative importance measure is not sensible if the spurious and core feature are overlapping, as in the second scenario. In this case, the shape's boundary is the ground-truth both for the spurious \textit{pattern} feature and the core \textit{shape} feature. Nevertheless, we compute this measure also for the second scenario.

To enable better comparison we again embed the metric within the effect estimation setting as in \cref{eq:ace_metric}, even though we do not expect an image without a watermark to apply any relevance within the bounding box. However, if it does, this might be a noteworthy finding.

\begin{align}\label{eq:ace_rma}
& ACE_{RMA} = \frac{1}{|M_\rho|\cdot |\mathcal{X}| }\sum_{m}^{M_{\rho}} \sum_{\mathrm{x,x'}}^{\mathcal{X}} RMA(\mathrm{x}) - RMA(\mathrm{x'})
\end{align}

Albeit this local approach reduces complexity, it has been noted that importance is hard to gauge when features have varying spatial extends \cite{Achtibat2022}. 
While it might give us a good estimate of the relevance of the watermark, the heatmap could still produce a wrong understanding of relative relevance for humans. 
In scenarios where the spurious feature is not spatially separated from the target feature, using region-specific metrics is harder. Here, one can only hope to estimate explained relative feature importance through proxies like prototypical samples as described in \cref{section:relmax_measure}.

\subsubsection{Relevance Rank Accuracy (RRA)}
The second metric adapted from \cite{Arras2022} is \textit{Relevance Rank Accuracy}.  
Relevance Rank Accuracy orders the input features (or pixels) by relevance and finds the $k$ most relevant pixels. The rank $k$ is equal to the size of the region of the feature, in question, here the boundary around the watermark. Computing the ratio of the top-k relevant pixels inside of the boundary $B$ to $k$ yields \textit{Relevance Rank Accuracy}:

\begin{align}
& P_{top-k}^{c}(\mathrm{x}) = (p_1, p_2,...,p_k | \ \  |R_{p_1,c}(\mathrm{x})| > |R_{p_2,c}(\mathrm{x})| > ... > |R_{p_k,c}(\mathrm{x})| ) \notag \\
& RRA(\mathrm{x}) = \sum_{c \in \ell} \frac{|P_{top-k}^{c}(\mathrm{x}) \cap B|}{|B|} * |R_c^{\ell}(\mathrm{x})|
\end{align}

Arguably, this metric loses more information on whether at least some importance is assigned to the spurious feature (watermark) than RMA and previous measures. It is possible that there is still some attribution to the watermark which is not accounted for, especially since in our benchmark $k$ is quite small. The difference in interpretation to RMA is that only slightly attributed pixels are ignored which is likely more in line to how humans interpret attribution maps. 
However, this metric ignores the relevance of the neurons in relation to each other, which therefore has to be reintroduced by separately weighing each concepts RRA value by the concepts magnitude of relevance $R_c^{\ell}$ again. 

\subsubsection{Pointing Game}
A step towards even more reduced complexity is the \textit{Pointing Game} metric, first introduced in \cite{Zhang2016}. The only difference between RRA and the Pointing Game metric is, that the latter \textit{binarizes} the question whether a feature is important by setting $k = 1$. If the most important pixel is inside the watermark bounding box $B$, the watermark is important for that filter, otherwise not.
It has been noted by others that the pointing game metric is sub-optimal for evaluating attribution maps, as pointing to the central pixel in an image already produces ''good'' results \cite{Gu2019}. Nevertheless, as in the watermark scenario the feature changes position on the sides of the image and is quite small, we think that this measure can still function as an approximation. 

\subsection{Relevance Maximization Measure}\label{section:relmax_measure}
The last measure we propose aims to specifically address the potentially \textit{glocal} explanation CRP provides. For the experiment we introduced in \cref{section:dataset_wdsprites}, the spurious feature, i.e. the watermark, is spatially separated from the core feature of the shape. Therefore it is reasonable to assume that by looking at a heatmap one can at least identify whether the spurious feature is important \textit{at all}, if not \textit{how} important. 
In real scenarios it is however often the case that spuriously correlated features are overlapping with the core feature or are not in one specific area. An example is the overall brightness of an image or the color or pattern on the object. Therefore, we introduced the second scenario where the spurious feature of a blurry or noisy pattern is overlapping with the target feature.

\begin{figure}[t!]
    \centering
    \includegraphics[width=0.6\textwidth]{thesis_latex_template/pics/compare_heatmap_pattern.png}
    \caption[Distinguish heatmaps for different patterns]{When a spurious feature overlaps with the core feature, local attribution maps do not help in identifying the truly important feature.}
    \label{fig:compare_heatmap_pattern}
\end{figure}

While the region-specific metrics should be able to identify relative feature importance for a spurious feature which is spatially separated from the core feature, this is not necessarily the case when they overlap. The mean attribution change measure could also show effects of intervention on a feature that is not easily localizable, but this effect might be difficult to interpret or even visually perceive (see \cref{fig:compare_heatmap_pattern}). The authors of CRP have shown prototype sets for individual concepts to help humans in identifying spurious features, because they potentially uncover not only the \textit{where} but also the \textit{what} of a concept's importance \cite{Achtibat2023}. So if we can find a way to measure which potentially overlapping feature is encoded by a reference set, it should be possible to quantify the humanly-readable relative feature importance even for these cases. 
We propose two different ways of gauging which feature a reference set encodes. 

The first measure, \textit{spurious feature share}, relates the number of images having the spurious feature to the number of images sharing the core feature. In an example of reference sets (\cref{fig:entangled_ref_set}) we give an intuition for this decision. Although there is relevance assigned to the spurious feature by the respective concept, it also seems to encode the shape. Differentiation is still somewhat possible for the first scenario, as we can clearly see attribution towards the watermark. For the second scenario (row 3 and 4) a human would not be able to recognize this concept as a pattern concept, because it could just as much be a shape concept. 
Achtibat et al. underline that relevance maximization is better than activation maximization at showing the concept within the context of its usage by the model. Therefore, if a watermark is mostly used when the ellipse shape is present, the reference set will mostly include images with watermark \textit{and} ellipse. This objective might be misguided in the case of spatially overlapping spurious features unless one at the same time has access to counterfactual examples where the same shape but different pattern is attributed less or negatively by that neuron. 

Our measure marks the concept shown in the first rows of \cref{fig:entangled_ref_set} positively as a \textit{watermark} concept. Yet, it would assign a much higher relative feature importance if there were also images showing watermarks and rectangles, i.e., if the shape feature was equally distributed. Adding to that, other latent factors might be just as distracting when they are encoded within the seen concepts, for example, the second set seems to mostly include ellipses in the bottom half of the image.

\begin{figure}[t!]
    \centering
    \includegraphics[width=
\textwidth]{thesis_latex_template/pics/rel_max_with_diversity_uncropped.png}
    \includegraphics[width=
\textwidth]{thesis_latex_template/pics/reference_set_overlap_example.png}
    \caption[Entangled Reference Sets]{Entangled Reference Sets of a concept of a model trained for the first scenario and for the second scenario.
    row 1: raw images selected in watermark case (with $W=1, S=1$), row 2: attribution maps of these, 
    row 3: raw images selected in pattern case (with $W=1, S=1$), row 4: attribution maps of these}
    \label{fig:entangled_ref_set}
\end{figure}

When one looks at the activation vs. relevance maximization sets in \cref{fig:act_rel_max}, an artifact pointing in that direction is apparent: 
The activation reference sets for a feature which seems to encode the watermark strongly, have mostly diagonally rotated rectangles on them. Our theory is that this is due to the lower edge of this shape resembling the lopes of the watermarks $W$. 
In contrast, the relevance maximization sets mostly show ellipses, possibly showing that the watermark is most relevant when the ellipse shape is present. 
This means that while in this specific trained model the watermark is the only decisive feature of the prediction (as of true model importance), the spurious feature is still connected to the shape or a concept for the shape was still learned. 
It is not clear whether CRP or even the underlying trained model are able to disentangle the latent factors enough to facilitate such distinctions.
To see whether activation maximization or relevance maximization with a maximum relevance instead of summed relevance target perform differently, we therefore include them into our analysis.
The approach principally works for any type of spuriously correlated feature, be it a watermark, color, pattern or background value. 

To combat the context issue we just described, another measure, \textit{class-specific spurious feature share} is introduced here. 
Relevance Maximization reference sets can be computed for any concept-conditioning set. 
By looking at each concept's class-specific reference sets, we essentially intervene on the class so the spurious feature can be studied more independently. If the watermark is indeed most strongly encoded by a concept, we expect reference images from either class to have it, while differing in shape.
In principal, the conditioning of reference sets could even be extended to the other latent factors rotation, scale and position. However, this would require manually selecting images which share the value for a latent factor and is not as automatic as CRP's extraction of class-specific relevances. 

\begin{align}\displaystyle
& \text{Spurious Feature Share: } \notag &  \\
& \mathcal{X}_{\star}^{c} = arg \sort_{\mathrm{x} \in \mathcal{X}} \mathcal{T}_{sum}^{c,rel}(\mathrm{x}) & \\
& \mathcal{X}_{k}^{c} = \{ \mathrm{x}_{1}^{\star},\mathrm{x}_{2}^{\star},..., \mathrm{x}_{k}^{\star} \} \subseteq \mathcal{X}_{\star}^{c} & \\
& Rel_c(w) = \sum_{\mathrm{x} \in \mathcal{X}_{k}^{c}:  W_{\mathrm{x}} = w } \frac{ R_c(\mathrm{x})   }
{max(|\{S_{\mathrm{x}} = s \}|}    & \\
& \text{SPF} = \sum_{c \in \ell} | Rel_c(1) - Rel_c(0) |  & \\
& \text{Class-Specific Spurious Feature Share } \notag & \\
& \mathcal{Y}(w)^{c,s} = | \{ \mathrm{x} \in \mathcal{X}_{k}^{c, S=s}: W_{\mathrm{x}} = w  \} | & \\
& \text{Y-SPF} = \sum_{c \in \ell} | \mathcal{Y}(1)^{c,1} - \mathcal{Y}(0)^{c,1} | +| \mathcal{Y}(1)^{c,0} - \mathcal{Y}(0)^{c,0} | * \tfrac{1}{2} & \\
\end{align}

TODO:

- show that average over models might be misleading i.e. some models have high feature importance yet do not encode the spurious feature within the reference sets at all!

\section{Comparison of Measures}\label{section:experiment_setup}
After having computed a ground truth effect of $\rho$ on the model feature importance $m_1$ and the effect on the explanation $m_2$, we can compare them and the measures under each other.
We have defined every measure in a way that it should be exactly 1 if all importance is assigned to the spurious feature. The scales of each measure are still technically not the same, but as they are all within the interval [0,1] and should, if correct, all follow the curve of the ground truth, we compare them in one plot for a first impression.
While a visual comparison of the curves of importance might award us an intuition to the measures overall reaction, tools like the mean squared error (MSE) can be applied to give us more precise insights. As Karimi et al. \cite{Karimi2023} already note,
it is not clear of which type (e.g. linear or non-linear) the relationship between $m_1$ and $m_2$ truly is. We do however expect the intervention effect to be at least linearly visible. 

Additionally, it is interesting to also look at how $m_1$ and $m_2$ relate independently of $\rho$, i.e., by measuring their correlation and looking at it with scatter plots. 
This will also hint at how the variance of importance both for $m_1$ and $m_2$ might differ in relation to the changing $\rho$. 


\section{CNN Model Zoo}\label{section:model_zoo}
To evaluate explanations, the model to test on can neither be to simple and therefore easy to explain, nor too large and therefore an overkill for the simple dataset at hand.
Through a simple trial-and-error search the architecture detailed in \cref{lst:cnnmodel} with three convolutional layers of eight channels each, one fully connected \textit{concept} layer with six neurons and finally the fully connected output layer with 2 output neurons was deemed most fitting for the task. While less convolutional channels or layers often resulted in the model not converging at all, having more neurons or potentially \textit{concepts} did not seem to add information but just redundancy and would have only increased computation time.
This model yields test accuracies over 90 percent when using the same feature distribution $\rho$ as used for training. As stated before, we fix all hyperparameters, as they are not part of this analysis, to values produced from a short search.
The only hyperparameter we deemed necessary to control for, is the random initialization of weights and biases. In preliminary experiments it became clear that some initializations react profoundly differently to $\rho$ than others. Other sources of randomness within the experiment are already fixed during data generation where we use a fixed seed to produce the noise distributions as well as the shuffling of samples for the training process. 
For further information on hyperparameters and training, refer to \cref{appendix:model}.


\chapter{Experimental Results}\label{chapter:results}

{ \color{red}
10-20 pages 

    \begin{itemize}
        \item (1/3 of thesis)
        \item whatever you have done, you must comment it, compare it to other systems, evaluate it
        \item usually, adequate graphs help to show the benefits of your approach
        \item caution: each result/graph must be discussed! what's the reason for this peak or why have you observed this effect
    \end{itemize}
}

\section{The Ground Truth Trained Models}
Here we report the general results of training 816 models with 16 different random initializations and 51 different values of $\rho$ (in 0.02 steps from 0 to 1).
The details on the model architecture and training process can be found in the \cref{appendix:model}. 
As already noted in the method section, the model is fairly small but still has high performance for the simple benchmark data we test it on. 

\begin{itemize}
    \item trained on 
\end{itemize}


\begin{figure}
    \centering
    \includegraphics{thesis_latex_template/pics/basic_accuracy.png}
    \caption[Accuracy]{Mean accuracy for models with the same coupling ratio $\rho$ (16 per bucket)}
    \label{fig:basic_accuracy}
\end{figure}


\begin{figure}
    \centering
    \includegraphics{thesis_latex_template/pics/accuracy_intervened.png}
    \caption[Accuracy for intervened Subsets]{Accuracy for models with the same coupling ratio $\rho$ (16 per bucket). As expected, for ellipses without and rectangles with the watermark, the accuracy drops as $\rho$ raises.}
    \label{fig:basic_accuracy}
\end{figure}

\begin{figure}
    \centering
    \includegraphics[width=\textwidth]{thesis_latex_template/pics/unnormalized_some_measures.png}
    \caption{Comparison of Measures Unnormalized}
    \label{fig:unnormalized}
\end{figure}


\begin{figure}
    \centering
    \includegraphics[width=\textwidth]{thesis_latex_template/pics/normalized_some_measures.png}
    \caption{Comparison of Measures Normalized to [0,1]}
    \label{fig:normalized}
\end{figure}


Interesting things i wanna say
\begin{itemize}
    \item Firstly: The reaction to the "bias" is way less strong than one might expect. Only at super high rates it actually becomes worth it to learn the spurious feature.
    \item However, that might be expected as that is what deep neural networks are specifically designed for
    \item in general, it is relieving to see that the importance of the biased feature for the model and the explanation is not completely departing from each other
    \item as the models feature importance rises with $\rho$, so does the explanation importance
    \item but it becomes clear that depending on the way the explanation is read / measured it can over- or under-emphasize the wm effect. 
    \item the seed makes a huge difference (maybe this is just because my model is not stable and it has to be recomputed???) 
    \item 
\end{itemize}

\section{Experiments}
\begin{itemize}
    \item what have I tried out with the different methods?
    \item list in concise order the possible measures
    \item ground-truth feature importance: mean logit change for output, phi correlation (= prediction flip)
    \item baseline explanation feature importance - thats what we compare to e.g. watermark bounding-box importance for summary heatmap
    \item special concept explanations feature importance
    \item how are the experiments set up, how do i make sure they are all well comparable
    \item to which other baseline could my measures be compared to?
\end{itemize}

\section{Results}
lots of plots!
\begin{itemize}
    \item what have I tried out with the different methods?
    \item what works and what doesn't
    \item plot for each experiment/possible method?
          \begin{itemize}
              \item watermark bounding box average relevance for different subgroups, somehow get difference
              \item variance of latent factors in relevance maximization image set - low variance means it encodes the concept
              \item naive: total relevance for watermark image region
              \item total activation + relevance of neuron given just watermark image
              \item one idea: take masked/bounding box approach again for neurons individual heatmaps
              \item nmf idea: somehow try to reduce the latent space to Watermark/Shape axis and measure variance in either direction
              \item centroids idea: use random DR algorithm and calculate ratio of centroid distances (needs latent factors again)
              \item causal idea??? somehow measure causal effect? - the other things are kind of causal or?
          \end{itemize}
\end{itemize}



\section{Evaluation}
\begin{itemize}
    \item evaluation of evaluation criteria:
    \begin{itemize}
        \item takes into consideration the whole latent space spanned by the concepts
        \item orients itself on known human cognition, user studies in this field would suggest this??? 
        \item performs similar to baseline watermark bounding box importance? 
        \item ...? \todo{define success criteria of finding a good measure}
    \end{itemize}
    \item which measure is the best according to those criteria
    \item which measure is the closest to ground truth
    \item which is the furthest from ground truth
    \item does measure find \textit{more information} than CRP itself and could possibly be used as a method on top of CRP for disentanglement/ spurious-core relation explanation?
    \item 
\end{itemize}


\section{Evaluating Method on Other Problems}\label{section:further_problems}

\section{Discussion}
\begin{itemize}
    \item do measures work
    \item what does causality help us with
    \item is CRP better for constant vector shift stuff or does it still suffer from it?
    \item can the application of those measures further explain/inform the explanation?
    \item what failed miserably
\end{itemize}

\chapter{Discussion}\label{chapter:discussion}
This thesis analyzed the fidelity of explained relative feature importance for a spurious feature. 
Our goal was to establish ways to measure the relative feature importance conveyed by the concept-based explanation method CRP.
Towards this aim we devised a causal framework of how a spurious feature can be coupled with a core feature in a data distribution. This enabled us to assess the true reaction of neural networks to such a coupling. Having yielded this ground-truth, we were able to compare it to the measured relative feature importance within the explanation. 
In the following we discuss the explanations' fidelity to the models. This discussion includes both quantifying how close the measures are to the explanation as well as qualitatively evaluating how informative the given explanation is. 

\section{Fidelity of Relative Feature Importance}
We recall the question central to our experiments:
\begin{quote}
    \textit{Is the relative feature importance of a spurious feature explained by CRP in correspondence with the true importance assigned to it by the model or is it more closely aligned to the associations within the data distribution or otherwise disturbed?}
\end{quote}

In summary our results show that the relative feature importance explained by CRP indeed follows the models true importance more closely than the linear coupling of the data distribution. 
It is however interesting to see that depending on the complexity of the measurement and the kind of spurious feature (watermark or pattern) results vary. 
Adding to that, the lack of comparability when only one model is explained means that the relative feature importance conveyed by the explanation might in practice be difficult to interpret.   

\begin{itemize}
    \item in general, it is relieving to see that the importance of the biased feature for the model and the explanation is not completely departing from each other
    \item as the models feature importance rises with $\rho$, so does the explanation importance
    \item due to the direct relationship between output (vector) and relevance vectors from the computation of LRP / CRP this is to be expected at least for rel and mac
    \item the coupling of importance however reduces, the more one looks at \textit{human understandable} or less complex measures of feature importance. 
    \item for multiple measures that ''lose information'' about the relevance of individual concepts, re-weighing them makes them way more accurate
    \item this is because an ''unimportant'' neuron might encode the spurious feature strongly, but overall the spurious feature has no strong importance
    \item here, the relative feature importance comes into play again. It is difficult for humans to decide whether the model is actually influenced by the spurious feature or not, if there are neurons encoding it, but their importance is maybe at like 10 percent
    \item therefore we need to look at heatmaps, rma etc. again in an ''unweighted'' way, by taking the importance change only of the 2 most important neurons for example
    \item but it becomes clear that depending on the way the explanation is read / measured it can over- or under-emphasize the wm effect. 
    \item 
\end{itemize}

\textit{Are We Explaining the Data or the Model?
Concept-Based Methods and Their Fidelity in Presence of Spurious Features Under a Causal Lense.}

\section{Ground-Truth Relative Feature Importance}
Although the ground-truth relative feature importance is merely a side-product necessary for our analysis, we have made compelling observations on it. In other work about explaining relative feature importance we have not found thorough descriptions of the non-linear dependency on the coupling of a spurious feature. It seems that the importance goes through a \textit{state change} when learning the spurious feature becomes easier than learning the core feature. From that point on, the model seems to only depend on the spurious features value. 
Nonetheless, even before this state change a slow raise of importance is visible. Our assumption is that there are two (or even three) strategies the models apply:
When the watermark is only weakly biased, the model completely ignores its value. The stronger the bias gets, the more the relative feature importance of the spurious feature is mixed with the core feature. In the last step the core feature is completely ignored. 
For realistic applications as Achtibat et al.'s \cite{Achtibat2022}  case of a real watermark, its explained relevance is often significantly below the relevance of core features. 
We therefore hypothesize that many real problems are located in the second stage, where both spurious and core features have some importance assigned. 
While it is not possible to differentiate these strategies even for the ground-truth importance, we consider their demarcation in explanations an important direction for further research.

\section{Complexity and Interpretability of Explanation}
Our analysis suggests that there is a trade-off between how closely the effect on the explanation follows the effect on the models and the complexity of measurement. 
When we measure the pixel- or neuron-wise differences, the effect of the intervention on our spurious feature is most accurately measured. 
However, as we have already hinted at during the construction of the measures, this effect is hard to gauge for humans. In the watermark scenario the spatial separation of spurious and core feature still somewhat enables an identification of which feature is more important when looking at attribution maps. For the pattern scenario though, where both features overlap, the attribution maps seem nearly identical or at least it is not obvious which feature is being used more. 
In our attempt to reduce and focus the complexity of the measured relative feature importance we constructed both region-specific measures as well as measures using the prototypical reference sets yielded through CRP. 

Interestingly, the region specific measures still show an effect of intervention for high values of $\rho$ in the pattern scenario. The most likely reason is that while all encoded concepts relevance lies mostly within the boundary of the shape for this case, the relevance of individual neurons still differs strongly. So a neuron that encodes the blurred pattern will assign high relevance to pixels within a blurred shape and low relevance to the same shape boundary with noisy pattern. 
Another possible explanation is less desirable: For some examples we found that attribution maps seemed to assign high relevance to pixels outside of the shape, when the pattern feature was not ''matching'' the neurons concept. This is not explaining the true reason for the decision and can be highly misleading. 


\subsection{Relevance Vectors}
- follows model importance the closest out of all measures
- this is not surprising as relevances are deterministically dependent on outputs through the backpropagation process
- also, because there are only 8 values instead of 8x64x64 like in the attribution maps, there is less chance for rounding errors or other disturbance to have an effect
- squared distance is closest to mlc (why is this working here but not in 64x64 case?)
- cosine distance also follows mlc very closely, but deviates more for high values of $\rho$ (is not getting as high, probably because of triangle inequality or something???) 
- for pattern scenario the values deviate way stronger than for watermark scenario
- this is to be expected: the pattern is way more entangled with the shape and therefore changing the pattern also more likely changes the explanation for the shape (e.g. if pixels on border of shape get blurred that might make it more \textit{ellipse-like})
- although the ''state-change'' at around 0.8 is still clearly visible for all measures, they overestimate the spurious feature's importance for lower values of $\rho$ and sometimes underestimate it for high values.
- most likely reason: same as above, but also that pattern feature seems to be harder to \textit{not learn}. Therefore it already plays a role for lower values? this is all kinda speculative... maybe I have to do some more tests and look at attribution maps to confirm this
- but in general it makes sense: because change of pattern happens around same area where truly important pixels are, it also has an effect on concepts that only encode shape


\subsection{Attribution Maps}
- For attribution maps, cosine distance is the closest to ground truth
- absolute distance overestimates spurious feature more, especially for pattern scenario
- but even for watermark scenario
- this is also to be expected as the absolute distance weighs small changes equally to big ones other than the squared or cosine distance
- In general, it is also not surprising that MAC works for somewhat for both cases
- for pattern scenario deviation is way stronger, probably due to earlier mentioned overlapping thingy for low values
- but also for high values: here, probably because for both values of W relevance still mostly lies within shape, so difference is smaller

\subsection{Relevance Mass Accuracy}
for watermark:
- deviate much stronger
- for low values it is ''accurate'' but for higher values it does not get as high
- this can be related to problem that Arras et al. \cite{Arras2022} mention themselves: it is not clear whether rma should even reach 1 or if it ever will as long as other features are present
- it actually follows mlc well, but just not at the same magnitude

for pattern:
- weighted rma is still somewhat close to gt
- rma of highest relevance neuron (for W=1) is very low and never gets above 0.4 even though pf is at almost 1, it even stagnates / gets worse
- definitly further away than for other scenario, showing that boundary thing doesn't work as well for this
- but expected would be that it does not make a difference at all?
- no actually not, same thing as before: one neuron encoding blurryness will have low relevance within shape for noise instances
- 
\subsection{Relevance Rank Accuracy}
for watermark:
- underestimates importance considerably
- this is probably the same as with rma, values of 1 are harder to obtain because that would mean 0 relevance in shape

for pattern:
- same as with rma: weighted version follows gt more closely than highest neuron version
- highest neuron version even gets less important for high values of $\rho$
- it is somewhat surprising that the top-k pixels are not always within the boundary region, the heatmaps must be wrong if pixels outside of the boundary region have highest relevance

\subsection{Pointing Game}
for watermark:
- both weighted and max pg overestimate importance for low(er) values of $\rho$. 
- two explanations: 
- 1. some neurons that have not learned anything useful are assigned relevance and by change have highest pixel in wm region
- 2. the explanation is misleading and assigns high relevance to watermark even when model is still able to somewhat ignore the watermark

for pattern:
- pointing game barely works
- explanation: the highest relevance pixel is always or most of the time inside the bounding box of the shape (as expected)
- therefore weighted one also underestimates for low values, where pattern still has some importance (is that only because models are badly trained?)

\subsection{Prototype Reference Sets}
for watermark:
- they all work a little, but stagnate around 0.2
- simple ref sets also works better than stats sets though
- so probably something is still wrong with stats sets???
- interestingly activation and relevance don't make such a big difference here

for pattern:
- works okay, i.e. seems to encode pattern to a degree that is human-interpretable
- by closer inspection: case where W=1 and S=0 or vice versa seems to not occur regardless for relevance sets. this means that it will still be hard to disentangle concepts by looking at reference sets
- means that especially for overlapping scenario humans will have difficulty assigning the proper feature to a reference set
- activation works better in this regard: probably because it doesnt do the ''in context of how its mostly used'' thing
- for stats thing: activation works pretty well
- relevance max works somewhat but not very well
- relevance sum doesn't work well at all
- possible explanation: all relevance sets encode pattern thingy, therefore w1 - w2 = 0 even for other shape
- it still has a tiny effect, but cant be considered proper


\begin{itemize}
    \item evaluation of evaluation criteria:
    \begin{itemize}
        \item takes into consideration the whole latent space spanned by the concepts
        \item orients itself on known human cognition, user studies in this field would suggest this??? 
        \item performs similar to baseline watermark bounding box importance? 
        \item ...? \todo{define success criteria of finding a good measure}
    \end{itemize}
    \item which measure is the best according to those criteria
    \item which measure is the closest to ground truth
    \item which is the furthest from ground truth
    \item does measure find \textit{more information} than CRP itself and could possibly be used as a method on top of CRP for disentanglement/ spurious-core relation explanation?
    \item 
\end{itemize}

\begin{itemize}
    \item do measures work
    \item what does causality help us with
    \item is CRP better for constant vector shift stuff or does it still suffer from it?
    \item can the application of those measures further explain/inform the explanation?
    \item what failed miserably
\end{itemize}

\section{Limitations and Outlook}


\newpage
%---------------------------------------------------
%----- Bibliography
%---------------------------------------------------
%\addcontentsline{toc}{chapter}{References}
%\renewcommand\bibname{References}
\bibliography{bibdata}


%---------------------------------------------------
%----- Appendix   
%---------------------------------------------------
%\backmatter

\changelocaltocdepth{1}
\appendix
\chapter{Appendix}\label{chapter:Appendix}


\section{Additional Details to LRP rules and implementation best practices}
\label{appendix:lrprules}
\begin{figure}[ht]
	\centering
	\label{fig:tesfigure}
	\includegraphics[width=\textwidth]{pics/test.png}
	\caption[Test Figure]{This is a test figure}
\end{figure}

\section{Preliminary Experiments}
\subsection{Plots}
\begin{figure}[ht]
	\centering
	\label{fig:blafigure}
	\includegraphics[width=\textwidth]{pics/test.png}
	\caption[Test Figure 2]{This is a test figure}
\end{figure}
\subsection{Causal Discovery on Neural Network Models Idea and Implementation?}


\section{Details on Model Architecture?}

\begin{lstlisting}{Python}
	self.convolutional_layers = nn.Sequential(
		nn.Conv2d(1, 8, kernel_size=3, stride=1, padding=0),
		nn.MaxPool2d(kernel_size=2, stride=2),
		nn.ReLU(),
		nn.Conv2d(8, 8, kernel_size=5, stride=1, padding=0),
		nn.MaxPool2d(kernel_size=2, stride=2),
		nn.ReLU(),
		nn.Conv2d(8, 8, kernel_size=7, stride=1, padding=0),
		nn.ReLU(),
	)
	self.linear_layers = nn.Sequential(
		nn.Linear(392, 6),
		nn.ReLU(),
		nn.Linear(6, 2),
	)
\end{lstlisting}

\section{Details on Adapted dSprites Dataset}\label{appendix:dsprites}
Has 737280 64x64 pixel binary images, only take rectangles and ellipses (Heart is to easy to distinguish?). so 419.. images.

The generating factors \verb|shape|, \verb|scale|, \verb|rotation|, \verb|x position| and \verb|y position| are known for each sample.

To adapt the benchmark for our purpose, only the first two shape classes (rectangle and ellipse) are used. A watermark in the form of a small \textit{w} was initially added to the lower-left corner of some images. During initial testing with only these adaptations it became clear that even the small convolutional neural network employed here is too powerful for this task as effectively dividing the image into two parts solves the problem and most neurons became irrelevant.
To make the spurious feature, which is the watermark \textit{w} more difficult to learn, its position is therefore varied across the edges of the image. Further a small uniform noise term is added to make the problem more realistic and the saliency maps more convincing and informative.
The aim of this new dataset is, to create the simplest possible scenario with known generating factors, while keeping it as realistic or close to real world application cases of attribution methods as possible. In \autoref{fig:dsprites_examples} the resulting images are visualized.



\todo{why the heck new dataset???}
\begin{itemize}
    \item why do we need another dataset for benchmarking watermark bias??
    \item some other benchmarks that deal with similar questions are...
    \item why am i not just using 3d shapes dataset? \url{https://github.com/deepmind/3dshapes-dataset/} (C. Burgess and H. Kim)
\end{itemize}

\section{Further Plots Groud Truth}


\end{document}
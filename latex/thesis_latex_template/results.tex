\chapter{Results}\label{chapter:results}

{ \color{red} 
\begin{itemize}
    \item (1/3 of thesis)
    \item whatever you have done, you must comment it, compare it to other systems, evaluate it
    \item usually, adequate graphs help to show the benefits of your approach 
    \item caution: each result/graph must be discussed! what’s the reason for this peak or why have you observed this effect
\end{itemize}
 }

\begin{enumerate}
    \item ground-truth feature importance: mean logit change for output, R2-score,  prediction flip 
    \item baseline explanation feature importance - thats what we compare to e.g. watermark bounding-box importance for summary heatmap
    \item special concept explanations feature importance 
    \item Test method on more complex dataset e.g. CLEVR-XAI
    \item compare CRP to other XAI methods?
\end{enumerate}


\section{Experiments} 
\begin{itemize}
    \item what have I tried out with the different methods? 
    \item what works and what doesn't
    \item plot for each experiment/possible method?
    \begin{itemize}
        \item watermark bounding box average relevance for different subgroups, 
        \\ somehow get difference
        \item variance of latent factors in relevance maximization image set - low variance means it encodes the concept
        \item naive: total relevance for watermark image region
        \item total activation + relevance of neuron given just watermark image
        \item one idea: take masked/bounding box approach again for neurons individual heatmaps
        \item nmf idea: somehow try to reduce the latent space to Watermark/Shape axis and measure variance in either direction
        \item centroids idea: use random DR algorithm and calculate ratio of centroid distances (needs latent factors again)
        \item causal idea??? somehow measure causal effect? - the other things are kind of causal or?
    \end{itemize}
\end{itemize}

\section{Discussion}

